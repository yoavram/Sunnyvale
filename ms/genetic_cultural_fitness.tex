\documentclass[12pt]{article} % Default font size of the document, change to 10pt to fit more text

\usepackage{geometry}
\geometry{
a4paper,
total={170mm,257mm},
left=20mm,
top=20mm,
headheight=12pt
}
\usepackage[parfill]{parskip} % Activate to begin paragraphs with an empty line rather than an indent
\usepackage{graphicx}
\usepackage{amssymb,amsmath,amsthm}
\usepackage{commath}
\usepackage{longtable}
\usepackage[dvipsnames]{xcolor}
\usepackage[unicode=true,colorlinks=true,urlcolor=CadetBlue,citecolor=black,linkcolor=black]{hyperref}
\PassOptionsToPackage{hyphens}{url} % url is loaded by hyperref
\usepackage[unicode=true]{hyperref}
%\usepackage{nameref} % included in hyperref
\usepackage{subcaption}

\usepackage{titlesec}
% \titleformat{command}[shape]{format}{label}{sep}{before}[after]
\titleformat{\section}{\normalfont\Large\bfseries}{\thesection}{1em}{}[]
\titleformat{\subsection}[runin]{\normalfont\large\bfseries}{\thesubsection}{1em}{}[]
\titleformat{\paragraph}[runin]{\normalfont\normalsize\bfseries}{\theparagraph}{1em}{}[]

% left margin	| space before (vertical) | space after (horizontal)
\titlespacing{\section}{0em}{0\baselineskip}{0\baselineskip}
\titlespacing{\subsection}{0em}{0\baselineskip}{1em}
\titlespacing{\paragraph}{0em}{0\baselineskip}{1em}
  
%SetFonts
% newtxtext+newtxmath
\usepackage{newtxtext} %loads helv for ss, txtt for tt
\usepackage{amsmath}
\usepackage[bigdelims]{newtxmath}
\usepackage[T1]{fontenc}
\usepackage{textcomp}
%SetFonts

% math commands
\newcommand*{\tr}{^\intercal}
\let\vec\mathbf
\newcommand{\matrx}[1]{{\left[ \stackrel{}{#1}\right]}}
\newcommand{\diag}[1]{\mbox{diag}\matrx{#1}}
\newcommand{\goesto}{\rightarrow}
\newcommand{\dspfrac}[2]{\frac{\displaystyle #1}{\displaystyle #2} }
\newtheorem{theorem}{Theorem}
\newtheorem{corollary}{Corollary}
\newtheorem{lemma}{Lemma}
\newtheorem{remark}{Remark}
\newtheorem{result}{Result}
\renewcommand\qedsymbol{} % no square at end of proof
\newcommand{\cl}{\mathbf{L}}
\newcommand{\cj}{\mathbf{J}}
\newcommand{\ci}{I}
\newcommand{\E}{\mathbf{E}}

% NatBib
\usepackage[round,colon,authoryear]{natbib}

% Title page
\title{Oblique Transmission with Genetic and Cultural Fitnesses}

% authors http://ftp.cc.uoc.gr/mirrors/CTAN/macros/latex/contrib/preprint/authblk.pdf
\usepackage[blocks,auth-lg,affil-it]{authblk}
\author{Yoav Ram}
\author{Marcus W. Feldman}
\affil{Department of Biology, Stanford University, Stanford, CA}

\date{\today}

% Document
\begin{document}
\maketitle

\section*{Model}

Consider a population with one trait and two trait values $A$ and $B$, with frequencies $x$ and $1-x$.
Phenotypes are transmitted either by learning from a parent with probability $\rho$ or by learning from a random adult model (e.g. teacher) with probability $1-\rho$.
The two phenotypes have distinct genetic ($w_A$ and $w_B$) and cultural ($v_A$ and $v_B$) fitness values~\citep{ElMouden2014}, defined by relative reproduction success and cultural influence, respectively.
\begin{equation} \label{eq:table}
\begin{array}{ccccc}
\hbox{ phenotype}&A&B\\
\hbox{ frequency}&x&1-x\\
\hbox{ genetic fitness}&w_A&w_B\\
\hbox{ cultural fitness}&v_A&v_B\\
\end{array}
\end{equation}

Given the frequency $x$ of phenotype $A$ in the current generation, the frequency in the next generation $x'$ is given by
\begin{equation} \label{eq:recurrence}
x' = \rho x \frac{w_A}{\bar w} + (1-\rho) x \frac{v_A}{\bar v} 
\end{equation}
where $\bar w = x w_A + (1-x) w_B$ is the population mean genetic fitness and $\bar v = x v_A + (1-x) v_B$ is the population mean cultural fitness. 
Note that if $v_A = v_B$ then \eqref{eq:recurrence} simplifies to 
$x' = \rho x \frac{w_A}{\bar w} + (1-\rho) x$,
which is eq.~2 from~\citet{Ram2018}.
So setting $v_A = v_B = \rho$ or even to so other parameter $v_A = v_B = \tau$ will not provide new insight.

\section*{Constant environment}

In a constant environment we will assume, without loss of generality, that $w_A > w_B$.
We will now find the equilibrium phenotype frequencies.
Setting $x'=x$ in~\eqref{eq:recurrence} and rearranging, we get a cubic polynomial $ax^3+bx^2+cx$.
%\begin{equation}
%\begin{aligned}
%G(x) &= a x^3 + b x^2 +c x, \quad \text{where} \\
%a &= (w_A - w_B) (v_A - v_B) \\ 
%b &= \rho(w_A v_B - w_B v_A) - (w_A - 2 w_B)(v_A - v_B) \\
%c &= -w_B(v_A - v_B) - \rho (w_A v_B - w_B v_A).  
%\end{aligned}
%\end{equation}
Two roots are easy to verify: $x=0 \Rightarrow x'=0$ and $x=1 \Rightarrow x'=\rho \frac{w_A}{w_A} + (1-\rho) \frac{v_A}{v_A} = 1$. 
Therefore, the cubic polynomial is $x (1-x) (x^*-x)$ where
\begin{equation}
%\begin{aligned}
x^* = 
%\frac{w_B v_B - (1-\rho) w_B v_A - \rho w_A v_B}{w_A v_A - w_A v_B - w_B v_A + w_B v_B} = \\
\rho \frac{w_B v_A - w_A v_B}{(w_A - w_B)(v_A - v_B)} - \frac{w_B}{w_A-w_B}.
%\end{aligned}
\end{equation}
%The conditions for $0 < x^* < 1$ are:
%\begin{equation}
%\begin{aligned}
%\frac{w_B (v_A - v_B)}{w_B v_A - w_A v_B}  < 
%\rho < 
%\frac{w_A (v_A - v_B)}{w_B v_A - w_A v_B} \\
%\end{aligned}
%\end{equation}

Now, $x=0$ is unstable if
\begin{equation}
%\frac{x'}{x} \eqsim 
\rho \frac{w_A}{w_B} + (1-\rho) \frac{v_A}{v_B} > 1,
\end{equation}
and $x=1$ is unstable if
\begin{equation}
%\frac{(1-x)'}{1-x} \eqsim 
\rho \frac{w_B}{w_A} + (1-\rho) \frac{v_B}{v_A} > 1.
\end{equation}

So both $x=0$ and $x=1$ are unstable if
\begin{equation} \label{eq:protected_poly_const}
w_B(v_B - v_A) < \rho (w_A v_B - w_B v_A) < w_A(v_B - v_A),
\end{equation}
in which case there exists a protected phenotypic polymorphism $0 < x^* < 1$.

There are three possibilities (disregarding equalities):
\begin{enumerate}
\item If $v_A > v_B$ and $\frac{w_A}{w_B} > \frac{v_A}{v_B}$ then condition~\eqref{eq:protected_poly_const} cannot apply as it requires $\rho<0$.
\item If $v_A > v_B$ and $\frac{w_A}{w_B} < \frac{v_A}{v_B}$ then condition~\eqref{eq:protected_poly_const} cannot apply as it requires $\rho>1$.
\item If $v_B > v_A$ then $w_A v_B - w_B v_A > 0$, $w_B(v_B - v_A)<w_A v_B - w_B v_A$, and $w_A v_B - w_B v_A > w_A(v_B - v_A)$ and condition~\eqref{eq:protected_poly_const} applies, ensuring a protected phenotypic polymorphism, if
\begin{equation}
\frac{w_B(v_B - v_A)}{w_A v_B - w_B v_A} < \rho < \frac{w_A(v_B - v_A)}{w_A v_B - w_B v_A},
\end{equation}
which can also be written as 
\begin{equation}
\frac{-s_v}{s_w-s_v} 
< \rho < 
(1+s_w) \cdot \frac{-s_v}{s_w-s_v},
\end{equation}
where $s_w=\frac{w_A}{w_B}-1>0$ and $s_v=\frac{v_A}{v_B}-1<0$.
\end{enumerate}

%\section*{Acknowledgements.}

%%%%%%%%%%%%%%%%%%%
\bibliographystyle{unsrtnat}
%\bibliography{/Users/yoavram/Documents/library}
\bibliography{genetic_cultural_fitness}
%%%%%%%%%%%%%%%%%%%
\end{document}  