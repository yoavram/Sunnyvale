\documentclass[14pt]{extarticle}
\usepackage{geometry}
\geometry{
a4paper,
total={170mm,257mm},
left=20mm,
top=20mm,
headheight=12pt
}

\usepackage[parfill]{parskip} % Activate to begin paragraphs with an empty line rather than an indent
\usepackage{graphicx}
\usepackage{amssymb,amsmath,amsthm}
\usepackage{commath}
\usepackage{longtable}
\usepackage[dvipsnames]{xcolor}
\usepackage[unicode=true,colorlinks=true,urlcolor=CadetBlue,citecolor=black,linkcolor=black]{hyperref}
\PassOptionsToPackage{hyphens}{url} % url is loaded by hyperref
\usepackage[unicode=true]{hyperref}
%\usepackage{nameref} % included in hyperref
\usepackage{subcaption}

\usepackage{titlesec}
% \titleformat{command}[shape]{format}{label}{sep}{before}[after]
\titleformat{\section}{\normalfont\Large\bfseries}{\thesection}{1em}{}[]
\titleformat{\subsection}[runin]{\normalfont\large\bfseries}{\thesubsection}{1em}{}[]
\titleformat{\paragraph}[runin]{\normalfont\normalsize\bfseries}{\theparagraph}{1em}{}[]

% left margin	| space before (vertical) | space after (horizontal)
\titlespacing{\section}{0em}{0\baselineskip}{0\baselineskip}
\titlespacing{\subsection}{0em}{0\baselineskip}{1em}
\titlespacing{\paragraph}{0em}{0\baselineskip}{1em}
  
%SetFonts
% newtxtext+newtxmath
\usepackage{newtxtext} %loads helv for ss, txtt for tt
\usepackage{amsmath}
\usepackage[bigdelims]{newtxmath}
\usepackage[T1]{fontenc}
\usepackage{textcomp}
%SetFonts

% math commands
\newcommand*{\tr}{^\intercal}
\let\vec\mathbf
\newcommand{\matrx}[1]{{\left[ \stackrel{}{#1}\right]}}
\newcommand{\diag}[1]{\mbox{diag}\matrx{#1}}
\newcommand{\goesto}{\rightarrow}
\newcommand{\dspfrac}[2]{\frac{\displaystyle #1}{\displaystyle #2} }
\newtheorem{theorem}{Theorem}
\newtheorem{corollary}{Corollary}
\newtheorem{lemma}{Lemma}
\newtheorem{remark}{Remark}
\newtheorem{result}{Result}
\renewcommand\qedsymbol{} % no square at end of proof
\newcommand{\cl}{\mathbf{L}}
\newcommand{\cj}{\mathbf{J}}
\newcommand{\ci}{I}
\newcommand{\E}{\mathbf{E}}

% NatBib
\usepackage[round,colon,authoryear]{natbib}

% Title page
\title{Oblique Transmission with Genetic and Cultural Fitnesses}

% authors http://ftp.cc.uoc.gr/mirrors/CTAN/macros/latex/contrib/preprint/authblk.pdf
\usepackage[blocks,auth-lg,affil-it]{authblk}
\author{Yoav Ram}
\author{Marcus W. Feldman}
\affil{Department of Biology, Stanford University, Stanford, CA}

\date{\today}

% Document
\begin{document}
\maketitle

\section*{Model}

Consider a population with one trait and two trait values $A$ and $B$, with frequencies $x$ and $1-x$.
Phenotypes are transmitted either by learning from a parent with probability $\rho$ or by learning from a random adult model (e.g. teacher) with probability $1-\rho$.
The two phenotypes have distinct genetic ($w_A$ and $w_B$) and cultural ($v_A$ and $v_B$) fitness values~\citep{ElMouden2014}, defined by relative reproduction success and cultural influence, respectively.
\begin{equation} \label{eq:table}
\begin{array}{ccccc}
\hbox{ phenotype}&A&B\\
\hbox{ frequency}&x&1-x\\
\hbox{ genetic fitness}&w_A&w_B\\
\hbox{ cultural fitness}&v_A&v_B\\
\end{array}
\end{equation}

Given the frequency $x$ of phenotype $A$ in the current generation, the frequency in the next generation $x'$ is given by
\begin{equation} \label{eq:recurrence}
x' = \rho x \frac{w_A}{\bar w} + (1-\rho) x \frac{v_A}{\bar v} 
\end{equation}
where $\bar w = x w_A + (1-x) w_B$ is the population mean genetic fitness and $\bar v = x v_A + (1-x) v_B$ is the population mean cultural fitness. 
Note that if $v_A = v_B$ then \eqref{eq:recurrence} simplifies to 
$x' = \rho x \frac{w_A}{\bar w} + (1-\rho) x$,
which is eq.~2 from~\citet{Ram2018}, so we hereafter assume that $v_A \ne v_B$.

\newpage

%%%%%%%%%%%%%%%%%%%%%%%%%%%%%%%%
\section*{Constant environment}
\subsection*{Equilibrium.}

In a constant environment we will assume, without loss of generality, that $w_A > w_B$.
To find the phenotype equilibrium frequencies,
set $x'=x$ in~\eqref{eq:recurrence} and rearrange to get a cubic polynomial $a_3x^3+a_2x^2+a_1x$.
%\begin{equation}
%\begin{aligned}
%G(x) &= a x^3 + b x^2 +c x, \quad \text{where} \\
%a &= (w_A - w_B) (v_A - v_B) \\ 
%b &= \rho(w_A v_B - w_B v_A) - (w_A - 2 w_B)(v_A - v_B) \\
%c &= -w_B(v_A - v_B) - \rho (w_A v_B - w_B v_A).  
%\end{aligned}
%\end{equation}
Two roots are easy to verify: $x=0$ and $x=1$
% \Rightarrow x'=0$ and $x=1 \Rightarrow x'=\rho \frac{w_A}{w_A} + (1-\rho) \frac{v_A}{v_A} = 1$. 
Therefore, the cubic polynomial is $x (1-x) (x^*-x)$ and the three roots are
\begin{align} \label{eq:roots}
&x = 0 \\
&x = 1 \\
&x^* = 
%\frac{w_B v_B - (1-\rho) w_B v_A - \rho w_A v_B}{w_A v_A - w_A v_B - w_B v_A + w_B v_B} = \\
%\rho \frac{w_B v_A - w_A v_B}{(w_A - w_B)(v_A - v_B)} - \frac{w_B}{w_A-w_B} = 
\frac{\rho s_w + (1-\rho) s_v}{s_w s_v},
\end{align}
where $1+s_w=\frac{w_A}{w_B}>1$ and $1+s_v=\frac{v_A}{v_B}>0$ (note this assumes that $s_w, s_v > -1$).
%Note that $x^*$ might be larger than 1 or less than 0, in which case it is not an equilibrium frequency.
%Note that $x^*<1$ iff $\rho \frac{w_A}{w_A-w_B} > (1-\rho) \frac{v_A}{v_B-v_A}$, 
%and $x^*>0$ iff $\frac{\rho}{1-\rho} > |\frac{s_v}{s_w}|$.
For $x^*$ to be a valid equilibrium frequency, it must be between 0 and 1, which implies the following conditions
\begin{equation} \label{eq:xstar_conditions}
\begin{cases}
0 < \rho < \frac{-s_v}{s_w - s_v},& \quad \text{if}\; s_w > 0 > s_v > -1,\\
0 < \rho < 1,& \quad \text{if}\; s_w > s_v > 0 \\
0 < \rho < \frac{s_v}{s_v - s_w},& \quad \text{if}\; s_v > s_w > 0
\end{cases}
\end{equation}

Now, $x=0$ is locally stable if 
\begin{equation}
\frac{x'}{x} \eqsim 
\rho (1+s_w) + (1-\rho) (1+s_v)
\end{equation}
is less than 1 and unstable if it is larger than 1.
Similarly, $x=1$ is locally stable if 
\begin{equation}
\frac{(1-x)'}{1-x} \eqsim 
\rho \frac{1}{1+s_w} + (1-\rho) \frac{1}{1+s_v}
\end{equation}
is less than 1 and unstable if it is larger than 1.
This implies the following conditions for the instability of both $x=0$ and $x=1$:
\begin{equation} \label{eq:fixation_instability_conditions}
\begin{cases}
\frac{-s_v}{s_w - s_v} < \rho < (1+s_w) \frac{-s_v}{s_w - s_v},& \quad \text{if}\; s_w > 0 > s_v > -1,\\
\text{n.a.},& \quad \text{if}\; s_w > s_v > 0 \\
\text{n.a.},& \quad \text{if}\; s_v > s_w > 0
\end{cases}
\end{equation}
 
Therefore, although a polymorphism fixed point exists~\eqref{eq:xstar_conditions}, it cannot be stable, because whenever it exists at least one of the fixations (in either $A$ or $B$) is stable~\eqref{eq:fixation_instability_conditions}.

%TODO - when is x=1 stable? When is x=0 stable?

\subsection*{Finite population.}
To model the effect of random drift due to a finite population size $N$, we use a Wright-Fisher model.
Let $X_t$ denote the number of $A$ individuals at generation $t$ ($X_t \in \{0, 1, \ldots, N\}$) so that $x=X_t/N$.
$x'$ still represents the frequency of $A$ in the next generation as in~\eqref{eq:recurrence}.
According to the Wright-Fisher model~\citep{Ewens2004}, the number of $A$ individuals in generation $(t+1)$ is binomially distributed:
\begin{equation}
X_{t+1} \sim Bin(N, x').
\end{equation}
The stochastic process $X_t$ has two absorbing states $X_t=0$ and $X_t=N$, corresponding to the two fixations in $A$ and $B$, respectively.
When one of the fixations is globally stable in the deterministic model~\eqref{eq:recurrence}, we are interested in the fixation probability $u(x)$ and time $T(x)$ starting from an arbitrary frequency $x$ and towards the stable absorbing state (either $x=0$ or $x=1$).
When a protected polymorphism exists, we are interested in the probabilities $u_0(x)$ and $u_1(x)$ for fixation towards $x=0$ or $x=1$, respectively, and the time for loss of polymorphism $T(x)$, or the fixation in either $x=0$ or $x=1$.

To proceed, we use a diffusion approximation, which requires that the differential selection (genetic or cultural) is small, or formally~\citep{Karlin1974a}
\begin{equation}
w_A - w_B \eqsim \frac{\eta_w}{N}, \quad v_A - v_B \eqsim \frac{\eta_v}{N},
\end{equation}
up to terms of order smaller than $1/N$.
Therefore, the infinitesimal mean displacement is
\begin{equation}
\mu(x)=\frac{1}{N}\E[X_{t+1}-X_t]=x(1-x) \eta
\end{equation}
where $\eta = \rho \eta_w + (1-\rho) \eta_v$.
The infinitesimal variance $\sigma^2(x)$ is
\begin{equation}
\sigma^2(x)=\frac{1}{N^2}Var[X_{t+1}-X_t]=x(1-x),
\end{equation}
up to terms of order smaller than $1/N$.

%We start with the case that $x=1$ is a stable equilibrium.
%, that is, either $\eta>0$ and $\eta_v>0$ or $\eta_w>0$, $\eta_v<0$, and $\rho > (1+s_w)\frac{-s_v}{s_w-s_v}$.
Therefore, the probability for fixation of $A$ (i.e. at $x=1$), starting with frequency $x$, is
\begin{equation} \label{eq:uA(x)}
u_A(x) = \frac{1-e^{-2 \eta x}}{1-e^{-2\eta}},
\end{equation}
and $u_B(x) = 1-u_A(x)$ is the probability for fixation of $B$ (i.e. at $x=0$).
The time to fixation at either $A$ or $B$ (starting with frequency $x$) is 
\begin{equation} \label{eq:T(x)}
T(x) = 
\frac{1-u(x)}{\eta} \int_0^x{\frac{e^{2 \eta z} - 1}{z(1-z)} dz} + 
\frac{u(x)}{\eta} \int_x^1{\frac{1-e^{-2 \eta (1-z)}}{z(1-z)} dz}
\end{equation}
where $u_A(x)$ is given in~\eqref{eq:uA(x)}, and $T(x)$ is measured in units of $1/N$.

For the fixation probability $u(x)$, we have the following result.
First, note that $u(x)$ is monotone increasing in $\eta$~\cite[see][Appendix E]{Ram2018}. 
Next, $\frac{d\eta}{d\rho} = \eta_w - \eta_v$ and therefore $u(x)$ is monotone increasing (decreasing) in $\rho$ when $\eta_w > \eta_v$ ($\eta_w < \eta_v$). 
That is, if the genetic fitness differential is bigger than the cultural fitness differential, then the probability of fixation increases with the vertical transmission rate; if the cultural fitness differential is bigger, than the fixation probability decreases with the vertical transmission rate.


We are specifically interested in the probability for stochastic fixation of $A$ (rather than $B$) starting at the (unstable) equilibrium polymorphism $x=x^*$ in~\eqref{eq:roots} (i.e. assuming $\eta_w>0$, $\eta_v<0$, and~\eqref{eq:fixation_instability_conditions}).

First, assuming $w_B \eqsim v_B \eqsim 1$ we have $\eta_w \eqsim s_w$ and $\eta_v \eqsim s_v$, so $\eta<0$ and $x^*=\eta / \eta_v \eta_w$~(from \eqref{eq:roots}) and we have
\begin{equation} \label{eq:uA(x*)}
u_A(x^*) = \frac{1-e^{-2 \eta^2/\eta_v \eta_w}}{1-e^{-2\eta}},
\end{equation}
Similarly, the time for loss of polymorphism when starting at the stable polymorphism $x=x^*$ under the same assumptions is
\begin{equation} \label{eq:T(x*)}
T(x^*) = 
\frac{1-u(x^*)}{\eta} \int_0^{x^*}{\frac{e^{2 \eta z} - 1}{z(1-z)} dz} + 
\frac{u(x^*)}{\eta} \int_{x^*}^1{\frac{1-e^{-2 \eta (1-z)}}{z(1-z)} dz}
\end{equation}



%\section*{Acknowledgements.}

%%%%%%%%%%%%%%%%%%%
\bibliographystyle{unsrtnat}
\bibliography{/Users/yoavram/Documents/library}
%\bibliography{genetic_cultural_fitness}
%%%%%%%%%%%%%%%%%%%
\end{document}  