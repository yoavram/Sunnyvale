\documentclass[14pt]{extarticle}
\usepackage{geometry}
\geometry{
a4paper,
total={170mm,257mm},
left=20mm,
top=20mm,
headheight=12pt
}

\usepackage[parfill]{parskip} % Activate to begin paragraphs with an empty line rather than an indent
\usepackage{graphicx} % Use pdf, png, jpg, or eps§ with pdflatex; use eps in DVI mode
% TeX will automatically convert eps --> pdf in pdflatex		

\usepackage{amssymb,amsmath,amsthm}
\usepackage{commath}
\usepackage{longtable}
\usepackage[hyphens]{url}
\usepackage[dvipsnames]{xcolor}
\usepackage[unicode=true,colorlinks=true,urlcolor=CadetBlue,citecolor=black,linkcolor=black]{hyperref}
\PassOptionsToPackage{hyphens}{url} % url is loaded by hyperref
\usepackage[]{authblk}

%SetFonts
% newtxtext+newtxmath
\usepackage{newtxtext} %loads helv for ss, txtt for tt
\usepackage{amsmath}
\usepackage[bigdelims]{newtxmath}
\usepackage[T1]{fontenc}
\usepackage{textcomp}
%SetFonts

% less space before sections 
% \@startsection {NAME}{LEVEL}{INDENT}{BEFORESKIP}{AFTERSKIP}{STYLE} 
%            optional * [ALTHEADING]{HEADING} 
\makeatletter
 \renewcommand\section{\@startsection {section}{1}{\z@}%
     {-2.5ex \@plus -1ex \@minus -.2ex}%
     {1.3ex \@plus.2ex}%
    {\Large\bfseries}}

% Yoav & Lee commands
\newcommand*{\tr}{^\intercal}
\let\vec\mathbf
\newcommand{\matrx}[1]{{\left[ \stackrel{}{#1}\right]}}
\newcommand{\diag}[1]{\mbox{diag}\matrx{#1}}
\newcommand{\goesto}{\rightarrow}
\newcommand{\dspfrac}[2]{\frac{\displaystyle #1}{\displaystyle #2} }
\newtheorem{theorem}{Theorem}
\newtheorem{corollary}{Corollary}
\newtheorem{lemma}{Lemma}
\newtheorem{remark}{Remark}
\newtheorem{result}{Result}
\renewcommand\qedsymbol{} % no square at end of proof
\newcommand{\cl}{\mathbf{L}}
\newcommand{\cj}{\mathbf{J}}
\newcommand{\ci}{I}

% NatBib
\usepackage[round,colon,authoryear]{natbib}

% Title page
\title{Vertical and oblique transmission with fluctuating transmission}

\author[a]{Yoav Ram}
\author[b]{Uri Liberman}
\author[a]{Marcus W. Feldman}
\affil[a]{Department of Biology, Stanford University, Stanford, CA}
\affil[b]{School of Mathematical Sciences, Tel Aviv University, Israel}

\date{\today}

% Document
\begin{document}
\maketitle

% Abstract
%\begin{abstract}
%\end{abstract}

% Introduction
\section*{Introduction}

Important modes of cultural transmission are vertical (from parent to offspring), horizontal (between peers), and oblique (from non-parental adults to offspring) transmission. 
\citet[ch.~3]{Cavalli-Sforza1981} first introduced models in which a specific trait is transmitted either vertically or obliquely.
Mathematical models of cultural evolution may include one or more modes of cultural transmission and in most of these models the mode of transmission by individuals is fixed; that is, it does not vary over time.
For example, \cite{Aoki2005} and \citet{McElreath2008} focused on competition between individual learning, innate learning, and social learning; \citet{Fogarty2017} compared the effects of different oblique mechanisms (random, success-biased, best-of-k, one-to-many) on the cultural richness and diversity of the population; and \citet{Aoki2012} modeled scenarios in which individual and social learning occur during separate stages in life.
In these studies, the cultural transmission rule did not have during the evolutionary process.

Several of these models included fluctuating selection, which could be due to environmental changes~\citep[reviewed in][]{Aoki2014}. 
We recently studied a model in which each individual can learn a dichotomous phenotype either from a parent, with probability $\rho$, or from a non-parental adult, with probability $1-\rho$ (\autoref{fig:transmission}).
We found that if selection fluctuates between favoring each of the two phenotypes, but on average favored both phenotypes for similar time periods, then a phenotypic polymorphism may be maintained~\citep{Ram2018}.
Furthermore, we found that if the environment changes very rapidly then lower $\rho$ values are likely to evolve, that is, oblique transmission is favored over vertical transmission.

There has been much less theory developed for fluctuating transmission.
Nevertheless, if we assume that social learning can be affected by ecological and demographic factors, such as the frequency of interactions~\citep{VanSchaik2003}, weather~\citep{Phithakkitnukoon2012}, population size or density~\citep{Fischer2015,Aureli1997a}, or stress~\citep{Farine2015}, then it is reasonable that the mode and rate of transmission will fluctuate over time and/or space.
For example, \citet{Webster2008} have demonstrated that minnows (small freshwater shoaling fish), increase their reliance on social learning when predation risk increases.
%For example, it may be the case that during cold seasons offspring are more likely to stay close to their parents and therefore learn vertically, whereas during warm seasons they are more likely to wander and interact with other adults and therefore learn obliquely.

Here, we analyze a model in which individuals learn from their parents with probability $\rho$ (vertical transmission), and from non-parental adults with probability $1-\rho$ (oblique transmission), where $\rho$ fluctuates over time or space~(\autoref{fig:transmission}).
We find that $\ldots$ (to be continued)  %TODO
We suggest that cultural evolution can be affected by environmental changes in both fitness and transmission fluctuations, and that further empirical and theoretical studies are required to disentangle these effects.

\begin{figure*}[h]
\centering
\includegraphics[width=0.5\linewidth]{../figures/{transmission}.png}
\caption{
\textbf{Cultural transmission with mixed vertical and oblique transmission.}
When a newborn matures, she copies her phenotype -- color -- from her mother with probability $\rho$, therefore becoming blue, or from some other female with probability $1-\rho$, in which case her color will depend on the frequency of blue and red adult females.}
\label{fig:transmission}
\end{figure*}

% Model & Results
\section*{Model \& Results}

Consider a very large population whose members are characterized by a single dichotomous cultural trait with phenotypes $A$ and $B$ and fitness values $w_A=1+s$ and $w_B=1$, respectively.
Phenotypes are transmitted vertically with probability $\rho$ or obliquely with probability $1-\rho$~(\autoref{fig:transmission}).
Given $x$ the frequency of phenotype $A$ at the current generation, the frequency of $A$ in the next generation is
\begin{equation} \label{eq:recurrence}
x' = \rho \frac{1+s}{\overline w} x + (1-\rho)x,
\end{equation}
where $\overline w = 1 + xs$ is the population mean fitness.

% Periodically fluctuating transmission
\subsection*{Periodically fluctuating transmission}

Suppose that the vertical transmission rate $\rho$ fluctuates, such that $\rho = \rho_1$ in odd generations and $\rho = \rho_2$ in even generations.
From \eqref{eq:recurrence}, the recurrence equations for two generations is then
\begin{equation}\begin{aligned} \label{eq:recurrence_two_generations}
x' = \rho_1 \frac{1+s}{\overline w} x + (1-\rho_1)x, \\
x'' = \rho_2 \frac{1+s}{\overline w} x' + (1-\rho_2)x'.
\end{aligned}\end{equation}

First, fixations of $A$ and $B$ ($x^*=1$ and $x^*=0$, respectively) are both equilibria, as they solve $x''=x$.
Second, when $s>0$ ($s<0$) because $x'/x = 1+\frac{\rho_1 s (1-x)}{1+sx} > 1$ ($<1$) and $x''/x' = 1+\frac{\rho_2 s (1-x')}{1+sx'} > 1$ ($<1$), the frequency of $A$ ($B$) increases every generation and $x^*=1$ ($x^*=0$) is globally stable.
Therefore, fluctuations in the mode of transmission ($\rho$) without fluctuations in selection lead to fixation of the favored phenotype, and there can be no stable polymorphisms.

% Periodically fluctuating transmission and selection
\subsection*{Periodically fluctuating transmission and selection}

Suppose that both the vertical transmission rate $\rho$ and the selection coefficient $s$ fluctuate together, so that when $A$ is favored the transmission rate is $\rho_A$, and when $B$ is favored the transmission rate is $\rho_B$.
The change in the frequency $x$ of phenotype $A$ when either $A$ or $B$ is favored is described by $F_A(x)$ and $F_B(x)$, respectively,
\begin{equation}\begin{aligned} \label{eq:recurrence_periodic_fluc}
F_A(x) = \rho_A \frac{1+s}{1+sx} x + (1-\rho_A)x, \\
F_B(x) = \rho_B \frac{1}{1+s-sx} x' + (1-\rho_B)x.
\end{aligned}\end{equation}

\paragraph{Symmetric periods -- $AkBk$.}
We first consider environments which fluctuate periodically every $k$ generations between favoring $A$ and $B$.
Using a linear approximation, fixations of $A$ ($x^*=1$) and $B$ ($x^*=0$) are locally stable if, respectively,
\begin{equation}\begin{aligned}
\big[F'_A(1) F'_B(1)\big]^k < 1, \\
\big[F'_A(0) F'_B(0)\big]^k < 1. \\
\end{aligned}\end{equation}
and if neither of these conditions are met then there exists a \emph{protected polymorphism}.
So for a protected polymorphism we require
\begin{equation}
1 < F'_A(1) F'_B(1) = \Big(1-\rho_A\frac{s}{1+s}\Big)\Big(1+\rho_B s\Big) 
= 1-\frac{1}{1+s}\big[\rho_B(1+s-\rho_A)-\rho_A\big],
\end{equation}
and
\begin{equation}
1 < F'_A(0) F'_B(0) = \Big(1+\rho_A s\Big)\Big(1-\rho_B \frac{s}{1+s}\Big) 
= 1+\frac{s}{1+s}\big[\rho_A(1+s) -\rho_B(1+\rho_A s)\big],
\end{equation}
which can be summarized as a condition on $\rho_B$,
\begin{equation} \label{eq:poly_condition_periodic_fluc}
\frac{\rho_A}{1+s(1-\rho_A)} < \rho_B < \frac{(1+s)\rho_A}{1+s\rho_A},
\end{equation}
or as a condition on the difference between $\rho_B$ and $\rho_A$
\begin{equation}
-s\rho_B(1-\rho_A) < \rho_B - \rho_A < s\rho_A(1-\rho_B).
\end{equation}
Surprisingly, the stronger the selection (i.e. the greater the value of $s$), the larger the difference in vertical transmission rates that allows a protected polymorphism.

\paragraph{Period of one -- $A1B1$.}
In the case of $k=1$ we can find $x^*$ the frequency of $A$ at the protected polymorphism.
Denote $F_B(F_A(x))-x = G(x) \cdot x (1-x) \cdot \gamma$ with
\begin{equation} \begin{aligned} \label{eq:xstar_periodic_fluc_k=1}
G(x) &= ax^2+bx+c, \\
\gamma &= -\frac{1 + sx}{s} \Big[ 1 + s + s^2 x (1-\rho)(1+x)\Big], \\
a &= s^2 \rho_A (1-\rho_A) (1 - \rho_B + \rho_B/\rho_A)) > 0, \\
b &= s(1-\rho_A)(2\rho_B - s\rho_A(1-\rho_B)), \\
c &= \rho_B - \rho_A - s\rho_A(1-\rho_B).
\end{aligned} \end{equation}
% $a+b+c>0$.
Then $x^*$ is a solution of the quadratic $G(x)=0$. %, given by $x^* = \frac{-b \pm \sqrt{b^2-4ac}}{2a}$.
The condition \eqref{eq:poly_condition_periodic_fluc}, which guarantees that $0 < x^* < 1$, is equivalent to $c<0$, and therefore it also guarantees that $\sqrt{b^2-4ac} > b$.
So, if $0<s<1$ (so that $b$ is guaranteed to be positive if LHS of eq.~\ref{eq:poly_condition_periodic_fluc}), then 
\begin{equation}
x^*= \frac{-b+\sqrt{b^2-4ac}}{2a}.
\end{equation}
\autoref{fig:rho1_rho2_phase_k=1} shows $x^*$, highlighting the area of the parameter space in which a protected polymorphism exists.
The figure demonstrates that the stronger the selection (x-axis), the greater the fluctuations in $\rho$ can be (y-axis) that still allow a polymorphic population (area between the dashed lines).

%\begin{align}
%G(0) = c > 0  \Leftrightarrow \rho_B > \frac{\rho_A(1+s)}{1+\rho_As}, \\
%G'(0) = b > 0 \Leftrightarrow \rho_B > \frac{\rho_A s}{2 + \rho_As}, \\
%G(1) = a + b + c = (1+s)[\rho_B - \rho_A + s \rho_B (1-\rho_A)], \\
%G'(1) = 2a + b = s(1-\rho_A)(2(1+s)\rho_B + s\rho_A(1-\rho_B)) > 0.
%\end{align} 

\begin{figure*}[htb]
\centering
\includegraphics[width=0.65\linewidth]{../figures/{rho1_rho2_phase_k=1}.pdf}
\caption{
\textbf{Protected polymorphism.}
The stable equilibrium of the frequency of phenotype $A$ \eqref{eq:xstar_periodic_fluc_k=1} for different selection coefficients ($s$ on x-axis) and size of fluctuations in vertical transmission rates ($\rho_B-\rho_A$ on y-axis) when both selection and transmission fluctuate every generation ($k=1$).
Dashed lines represent $\rho_B=\frac{\rho_A}{1+s(1-\rho_A)}$ and $\rho_B=\frac{(1+s)\rho_A}{1+s\rho_A}$, the limits on $\rho_B-\rho_A$ from~\eqref{eq:poly_condition_periodic_fluc} that allow protected polymorphism.
Here, $\rho_A=0.5$.}
\label{fig:rho1_rho2_phase_k=1}
\end{figure*}

\paragraph{Asymmetric period -- $AkBl$.}
More generally, phenotype $A$ could be favored for $k$ generations and phenotype $B$ for $l$ generations, and the transmission rate follows the same cycle with $\rho=\rho_A$ when $A$ is favored and $\rho=\rho_B$ when $B$ is favored.
The requirements for a protected polymorphism are now
\begin{equation}\begin{aligned}
1 &< (F'_A(0))^k (F'_B(0))^l = a^k \big(b - \Delta \rho s/(1+s)\big)^l,\\
1 &< (F'_A(1))^k (F'_B(1))^l = b^k (a + \Delta \rho s)^l,
\end{aligned}\end{equation}
where $a=1+\rho_A s>1$, $b=1-\rho_A\frac{s}{1+s}<1$, and $\Delta \rho = \rho_B - \rho_A$.
This leads to a condition similar to eq.~20 in~\citet{Ram2018}, but more complex due to the inclusion of $\Delta \rho$:
\begin{equation} \label{eq:poly_condition_periodic_fluc_k_l}
\frac{-\log{b}}{\log{\big(a + s \Delta \rho \big)}} < 
\frac{l}{k} < 
\frac{\log{a}}{-\log{\big(b - s \Delta \rho/(1+s)\big)}}.
\end{equation}
Therefore, for a given value of $\rho_A$, increasing the vertical transmission rate specifically when $B$ is favored decreases the environmental period ratio $l/k$ that permits protected polymorphism; decreasing $\rho_B$  will have the opposite effect, increasing the ratio $l/k$ that permits a protected polymorphism.

% Randomly fluctuating transmission and selection
\subsection*{Randomly fluctuating transmission and selection}

Now suppose that both selection and transmission fluctuate randomly.
Rewrite eq.~28 from \citet{Ram2018} so that $\rho$ is also a random variable
\begin{equation}
x_t = x_t \frac{1 + \rho_t s_t + x_t (1 - \rho_t) s_t}{1 + x_t s_t},
\end{equation}
where $s_t$ are i.i.d (independent and identically distributed), $Pr(-1+C<s_t<D)=1$ for some positive $C$ and $D$, $\rho_t$ are i.i.d, and  $0<\rho_t<1$ ($t=0,1,2,\ldots$).
Therefore, $z_t = \rho_t s_t$  are independent and identically distributed and $P(-1+C < z_t < D)$. 
From \citet[][results~6 and 7]{Ram2018} we have:
\begin{itemize}
\item Suppose $E[log(1+\rho_t s_t)]>0$. Then $x^*=0$ is not stochastically locally stable and in fact $P(lim_{t \to \infty} x_t=0) = 0$, i.e., fixation of $B$ almost surely does not occur.
\item Suppose $E[log(1+\rho_t s_t)]<0$. Then $x^*=0$ is stochastically locally stable. 
\item Similarly, if $E[log(1-\rho_t s_t/(1+s_t)]<0$, then $x^*=1$ is stochastically locally stable, and if $E[log(1-\rho_t s_t/(1+s_t))]>0$, then fixation of $A$ almost surely does not occur.
\item In particular, if $E[\rho_t s_t] = cov(\rho_t, s_t) + E[\rho_t] E[s_t] \le 0$ then $x^*=0$ is stochastically locally stable, and similarly if $E[-\rho_t s_t/(1+s_t)] = cov(\rho_t, -s_t/(1+s_t)) - E[\rho_t] E[s_t/(1+s_t)] \le 0$ then $x^*=1$ is stochastically locally stable.
\item It is not possible that $E[log(1+\rho_t s_t)]$ and $E[log(1-\rho_t s_t/(1+s_t)]$ are both negative, as their sum is positive
\begin{multline}
E[log(1+\rho_t s_t)] + E[log(1-\rho_t s_t/(1+s_t)] = \\
E[log(1+\rho_t s_t) + log(1-\rho_t s_t/(1+s_t)]= \\
E[log\big((1+\rho_t s_t)(1-\rho_t s_t/(1+s_t)\big)]= \\
E[log\big( 1+\rho_t(1-\rho_t)s_t^2/(1+s_t) \big)] > 0,
\end{multline}
and therefore it is not possible that both fixations are stochastically locally stable.
\end{itemize}

\paragraph{Examples.}
First, if $s_t$ and $\rho_t$ are independent ($cov(s_t, \rho_t)=0$) and $s_t$ is symmetric around zero, then $E[\rho_t s_t]=0$ and $E[-\rho_t s_t/(1+s_t)]>0$ (because $E[s_t/(1+s_t)] < E[s_t]$).
Therefore, fixation of $B$ is stochastically locally stable and fixation of $A$ almost surely does not occur.
For example, let $s_t \sim U(-1, 1)$ and $\rho_t \sim U(0,1)$ independently (in particular, $cov(\rho_t, s_t)=0$), then $E[log(1+\rho_t s_t)]\approx -0.07315$ and $E[log(1-\rho_t s_t/(1+s_t))]\approx 0.2337$.

However, note that symmetry of $s_t$ around zero provides an advantage to phenotype $B$: using Jensen's inequality, $E[w_A/w_B] = E[1+s_t] = 1 \le E[1/(1+s_t)] = E[w_B/w_A]$.
Therefore, if we take the i.i.d fitness random variables for $A$ and $B$ to be $w_{A,t}, w_{B,t} \sim U(0,1)$, respectively, and define $s_t=(w_{A,t}-w_{B,t})/w_{B,t}$, then neither $A$ nor $B$ has an advantage, on average (i.e. $E[w_{A,t}/w_{B,t}]=E[w_{B,t}/w_{A,t}]$, see \autoref{fig:beta}A), and both $E[\rho_t s_t]$ and $E[-\rho_t s_t/(1+s_t)]$ are positive, so that both fixations are not stochastically locally stable, and we expect the population to converge to a polymorphic distribution.

Second, if $s_t$ and $\rho_t$ are not independent such that $cov(\rho_t, s_t) \ne 0$, there can be fixation.
Let $w_{A,t}, w_{B,t} \sim U(0,1)$, $s_t=(w_{A,t}-w_{B,t})/w_{B,t}$ and $\rho_t \sim Beta(1+s_t, 1)$ (a beta distribution with parameters $1+s_t$ and $1$). 
The covariance of $s_t$ and $\rho_t$ is positive ($cov(s_t, \rho_t) \approx 4$; estimated by averaging $10^8$ random values of $s_t$ and $\rho_t$); that is, vertical transmission is more likely when $A$ is favored (i.e. $s_t>0$) and oblique transmission is more  likely when $B$ is favored (i.e. $s_t<0$; \autoref{fig:beta}).
Then $E[log(1+\rho_t s_t)] >0$ and $B$ almost surely doesn't fix. 
Also, $E[log(1-\rho_t s_t / (1+s_t)] < 0$, so fixation of $A$ is stochastically locally stable.
The opposite occurs if $\rho_t \sim \beta(1, 1+s_t)$ and the covariance of $s_t$ and $\rho_t$ is negative ($cov(s_t, \rho_t) \approx -4$). In that case, fixation of $B$ is stochastically locally stable and $A$ almost surely doesn't fix.

\begin{figure*}[hbt]
\centering
\includegraphics[width=\linewidth]{../figures/{beta}.png}
\caption{
\textbf{Covariance of selection and transmission.}
\textbf{(A)} Histogram of $w_{A,t}/w_{B,t}$ where $w_{A,t}$ and $w_{B,t}$ are identically and independently distributed uniform random variables $U(0,1)$.
\textbf{(B)} Histogram of $s_t = (w_{A,t}-w_{B,t})/w_{B,t}$.
\textbf{(C)} Histogram of $\rho_t \sim Beta(1+s_t, 1)$.
\textbf{(D)} The joint distribution of $\rho_t$ and $s_t$ demonstrates a positive correlation $cov(s_t, \rho_t)>0$.
}
\label{fig:beta}
\end{figure*}

Third, it is also possible that both fixations are not stochastically locally stable even if $s_t$ and $\rho_t$ covary, but, as in the case of periodic fluctuations, this can only occur if the fluctuations in $\rho_t$ are small.
For example, \autoref{fig:rho1_rho2_stoch_p} shows the expected outcome when $s_t=s$ and $\rho_t=\rho_1$ with probability $p=0.505$, while $s_t=-s$ and $\rho_t=\rho_2$ with probability $1-p=0.495$.
The blue and red areas show expected fixation of $A$ and $B$, respectively (i.e. stochastic local stability) and the white area shows expected protected polymorphism (i.e. neither fixation is stochastically locally stable). 

\begin{figure*}[hbt]
\centering
\includegraphics[width=0.75\linewidth]{../figures/{rho1_rho2_stoch_p}.pdf}
\caption{
\textbf{Stochastic local stability.}
Here, $s_t=0.05$ and $\rho_t=\rho_1$ with probability $p=0.505$ and $s_t=-0.05$ and $\rho_t=\rho_2$ with probability $1-p=0.495$.
The diagonal represents the case of no transmission fluctuations; \citet[Fig.~2]{Ram2018} demonstrated that with a constant transmission rate $\rho=0.1$ and the above distribution of $s_t$, neither fixation is stochastically stable.
}
\label{fig:rho1_rho2_stoch_p}
\end{figure*}

% TODO full correlation between rho and s?

% Finite population size
\subsection*{Finite population size}

To include the effects of random genetic drift due to finite population size in the above deterministic model, we follow \citet{Ram2018} and develop a diffusion equation approximation.
In~\citet{Ram2018} only selection fluctuated via $s_t$, but here we also have transmission fluctuating via $\rho_t$. 

We obtain a result similar to result 11 from~\citet{Ram2018}:
The mean $\mu(x)$ and variance $\sigma^2(x)$ of the change in the frequency $x$ of phenotype $A$ in the case of a cycling environment $AkBl$, where $k+l=n$, are
\begin{equation} \label{eq:drift_diffusion_terms}
\mu(x) = S_n x(1-x)
\quad \text{and} \quad
\sigma^2(x) = n x (1-x),
\end{equation}
where $S_n = \sum_{t=1}^{n}{\rho_t s_t}$.
Furthermore, combining \eqref{eq:drift_diffusion_terms} with eqs.~46-47 from \citet{Ram2018}, we find that the fixation probability of phenotype $A$, starting from a frequency $x$ of phenotype $A$, is
\begin{equation}
u(x) = \frac{1 - e^{-2 \frac{S_n}{n} x}}{1 - e^{-2 \frac{S_n}{n}}}.
\end{equation}
From result 10 in \citep{Ram2018}, $u(x)$ is monotone increasing in $S_n/n$ which is the average selection coefficient of $A$ weighted by the vertical transmission rates $\rho_t$.
Therefore, if $s_t$ and $\rho_t$ are positively (negatively) correlated, $S_n/n$ increases (decreases), and the fixation probability $u(x)$ increases (decreases). 
This occurs because selection only affects those individuals that transmit their phenotype to their own offspring (i.e. vertically), and a fraction $1-\rho_t$ of the population individuals is effectively masked (for better or worse) from selection at each generation.

% Fluctuations in space
\subsection*{Fluctuations in space}

We now describe a model in which fluctuations in selection and transmission occur in space, rather then time; that is, we model a population divided to two demes.
Selection (e.g. reproduction) and transmission (e.g. learning, development) occur within the demes, followed by migration of sub-adults -- individuals that already acquired their phenotype but have yet to reproduce.
The frequency of phenotype $A$ in deme $j$ is denoted by $x_j$, and therefore after selection and transmission the frequencies are 
\begin{equation} \label{eq:migration_model_selection_transmission}
x_j' = \rho_j \frac{w_j}{\overline{w}_j} x_j + (1-\rho_j) x_j,
\end{equation}
where $w_j$ is the fitness of phenotype $A$ in deme $j$ relative to the fitness of phenotype $B$; $\overline{w}_j=w_j x_j + (1-x_j)$ is the population mean fitness in deme $j$; and $\rho_j$ is the vertical transmission rate in deme $j$.

Following migration, the frequencies of $A$ in the two demes are
\begin{equation} \label{eq:migration_model}
\begin{aligned}
x_1'' &= (1-m_1) x_1' + m_2 x_2', \\
x_2'' &= m_1 x_1' + (1-m_2) x_2',
\end{aligned}
\end{equation}
where $0 \le m_j \le 1/2$ is the migration rate out of deme $j$. 

\paragraph{Unconditional migration and symmetric selection.}
 
For the simple case of unconditional migration $m_1=m_2=m$ \citep{McPeek1992} and symmetric selection $w_1=1/w_2=w>1$, the recursions~\eqref{eq:migration_model} become
\begin{equation}\begin{aligned} \label{eq:migration_model_unconditional_symmetric}
x_1'' &= (1-m)x_1\Big(\rho_1 \frac{w}{\overline w_1} x_1 + 1-\rho_1 \Big) + m x_2\Big(\rho_2 \frac{1/w}{\overline w_w} x_2 + 1-\rho_2 \Big), \\
x_2'' &= m x_1\Big(\rho_1 \frac{w}{\overline w_1} x_1 + 1-\rho_1 \Big) + (1-m) x_2\Big(\rho_2 \frac{1/w}{\overline w_w} x_2 + 1-\rho_2 \Big).
\end{aligned}
\end{equation}

We have the following results:
\begin{itemize}
\item With only oblique transmission ($\rho_1=\rho_2=0$), the only equilibria are the neutral equilibria $(x^*,x^*)$ for any $0 \le x^* \le 1$.
\item With only vertical transmission ($\rho_1=\rho_2=1$), the fixation equilibria  $(0,0)$, $(1,1)$ are unstable and there exists a protected polymorphism
\begin{equation}
\begin{aligned} \label{eq:migration_model_unconditional_symmetric_polymorphism}
x_1^* &= \frac{w-1-m(w+1) + \sqrt{\Delta}}{2(w-1)}, \\
x_2^* &= 1-x_1^*,
\end{aligned}
\end{equation}
where $\Delta = m^2(w+1)^2+(1-2m)(w-1)^2$.
\item With only vertical transmission in one deme ($\rho_1=1$) and a combination of both vertical and oblique transmission in the other deme ($\rho_2=\rho$), fixation of $B$ is unstable, and fixation of $A$ is stable if and only if the vertical transmission rate in deme 2 is below $\hat \rho$; that is,
\begin{equation} \label{eq:migration_model_unconditional_symmetric_condition_rho}
\rho < \hat \rho = \frac{m}{m+(1-m)(w-1)},
\end{equation}
or if the migration rate is above $\hat m$; that is,
\begin{equation} \label{eq:migration_model_unconditional_symmetric_condition_m}
m > \hat m = \frac{\rho (w-1)}{\rho (w-1) +1-\rho}.
\end{equation}
\end{itemize}

\paragraph{Proof.}

For the stability of the equilibria when $\rho_1=\rho_2=1$, the characteristic polynomial of the Jacobian is the same for both fixation equilibria $(0,0)$ and $(1,1)$ and is given by $a \lambda^2 + b \lambda + c$ with
\begin{equation}
\begin{aligned}
a &= 1, \quad
b &= -\frac{(1-m)(w^2+1)}{w}, \quad
c &=  1-2m.
\end{aligned}
\end{equation}
Since $a>0$, $b<0$, and $c>0$, and because $a+b+c=-\frac{1-m}{w}(w-1)^2<0$, the leading eigenvalue is greater than one, both equilibria are unstable, and a protected polymorphism exists and is given by~\eqref{eq:migration_model_unconditional_symmetric_polymorphism}.

When $\rho_1=1$ and $\rho_2=\rho$, the stability of $(0,0)$ (i.e. fixation in $B$) is determined by the characteristic polynomial with coefficients
\begin{equation}
\begin{aligned}
a &= 1, \quad
b &= -\frac{(1-m)(w+1-\rho+\rho/w)}{w}, \quad
c &=  (1-2m)(w(1-\rho)+\rho).
\end{aligned}
\end{equation}
Again, $a>0$ and $c>0$ and stability can be analyzed by the sign of $a+b+c$.
For $\rho=1$ we already have $a+b+c<0$.
For $\rho=0$ we have $a+b+c=-m(w-1)<0$.
Finally, $a+b+c$ is a linear function of $\rho$ and therefore $a+b+c<0$ for any valid $\rho$, the leading eigenvalue is greater than one, and $(0,0)$ is unstable.

The stability of $(1,1)$ (i.e. fixation in $A$) is determined by the characteristic polynomial with coefficients
\begin{equation}
\begin{aligned}
a &= 1, \quad
b &= -\frac{1-m}{w}(1 + w(1-\rho) +w^2 \rho), \quad
c &= \frac{1-2m}{w} (w\rho+1-\rho),
\end{aligned}
\end{equation}
and real roots. 
Again, $a>0$, $b<0$, and $c>0$, and if $\rho=1$ then $a+b+c<0$, as before.
However, for $\rho=0$ we have $a+b+c = \frac{m}{w}(w-1)>0$, in which case the stability is determined by $2a+b=\frac{m(w+1)+w-1}{w}$, the derivative of the polynomial at $\lambda=1$, which is positive, and therefore both eigenvalues occur in $(0,1)$ and fixation is stable.

For any $\rho$, $a+b+c>0$ if and only if
\begin{equation}
\rho < \hat \rho = \frac{m}{m+(1-m)(w-1)},
\end{equation}
and then $(1,1)$ is stable if and only if $2a+b>0$.
We have already seen that when $\rho=0$ we have $2a+b>0$.
When $\rho = \hat \rho$, we have $2a+b=\frac{m(w+1)+w-1}{w(1-m)(w-1)}>0$.
Since $2a+b$ is linear in $\rho$, we find that $2a+b>0$ for $\hat \rho \ge \rho \ge 0$ and therefore \eqref{eq:migration_model_unconditional_symmetric_condition_rho} guarantees the stable fixation of $A$.

The values of $a+b+c$ at $m=0$ and $m=1$ are $-\frac{\rho(w-1)^2}{w}<0$ and $\frac{(1-\rho)(w-1)}{w}>0$, respectively, and $a+b+c$ is linear in $m$, so that we have $a+b+c>0$ when 
\begin{equation}
m > \hat m = \frac{\rho (w-1)}{\rho (w-1) +1-\rho}.
\end{equation}
In addition, $2a+b$ is linear in $m$ and equal to 2 when $m=1$; it is equal to zero when $m=(w-1)(w\rho-1)/(w(w-1)\rho+w+1) < \hat m$, and therefore \eqref{eq:migration_model_unconditional_symmetric_condition_m} is sufficient for stability of fixation of $A$.

\paragraph{Examples.}

\autoref{fig:migration_rho} and \autoref{fig:migration_m} show the stable frequencies of phenotype $A$~ (eq.~\ref{eq:migration_model_unconditional_symmetric_polymorphism}) and the stable population mean fitnesses in the two demes with symmetric selection such that $A$ is favored in deme 1 and $B$ is favored in deme 2 with similar selection intensities.
Notably, in the absence of oblique transmission (\autoref{fig:migration_m}, left column), migration decreases the differences between the demes and reduces the population mean fitnesses.
With some oblique transmission, but equal in both demes, results are similar (not shown).
However, if oblique transmission is stronger in deme 2 than in deme 1 (\autoref{fig:migration_m}, middle and right columns), the stable frequency of $A$ increases in both demes.
Therefore, the mean fitness in deme 1 decreases to a lesser extent than in deme 2, and even increases when the migration rate is high enough.

The polymorphism $(x_1^*, x_2^*)$ (eq.~\ref{eq:migration_model_unconditional_symmetric_polymorphism}) is protected when transmission rates are equal, but not when transmission rates differ enough and migration is strong enough, in which case fixation of phenotype $A$ is stable. 
For example, with enough oblique transmission ($\rho_2<\hat \rho$) in deme 2, phenotype $A$ fixes, and the stronger migration is, the less oblique transmission is required to fix $A$, see shaded areas in \autoref{fig:migration_rho}.
Similarly, with enough migration ($m > \hat m$), phenotype $A$ fixes, and the more oblique transmission in deme 2, the less migration is needed to fix $A$; see shaded area in \autoref{fig:migration_m}.

\begin{figure*}[ht]
\centering
\includegraphics[width=0.75\linewidth]{../figures/migration_rho.pdf}
\caption{
\textbf{Oblique transmission and migration: effect of transmission.} 
The figure shows $x^*_i$ the stable frequencies of $A$ (top row) and $\bar{w}^*_i$ the population mean fitnesses (bottom row) in the two demes.
Selection is symmetric between the two demes (the fitness of phenotype $A$ is $w_1=2$ in deme 1 and $w_2=0.5$ in deme 2; the fitness of phenotype 2 is $1$ in both demes).
The vertical transmission rate is $\rho_1=1$ in deme 1, and $\rho_2$ (x-axis) in deme 2.
Migration rate $m$ is 0.05, 0.1, or 0.25 I the left, middle, and right columns, respectively.
The shaded area denotes stables fixation of phenotype $A$ according to eq.~\ref{eq:migration_model_unconditional_symmetric_condition_rho}.
Lines are drawn by iterating eq.~\ref{eq:migration_model_unconditional_symmetric} until frequencies in consecutive generations differ by less than $10^{-4}$, starting with equal frequencies.
}
\label{fig:migration_rho}
\end{figure*}

\begin{figure*}[ht]
\centering
\includegraphics[width=0.75\linewidth]{../figures/migration_m.pdf}
\caption{
\textbf{Oblique transmission and migration: effect of migration.} 
The figure shows $x^*_i$ the stable frequencies of $A$ (top row) and $\bar{w}^*_i$ the population mean fitnesses (bottom row) in the two demes.
Selection is symmetric between the two demes (the fitness of phenotype $A$ is $w_1=2$ in deme 1 and $w_2=0.5$ in deme 2; the fitness of phenotype 2 is $1$ in both demes).
The vertical transmission rate is $\rho_1=1$ in deme 1, and $\rho_2=1$, $0.4$, and $0.2$, in the left, middle, and right columns, respectively, in deme 2.
Migration rate $m$ is on the x-axis.
The shaded area denotes stables fixation of phenotype $A$ according to eq.~\ref{eq:migration_model_unconditional_symmetric_condition_m}.
Lines are drawn by iterating eq.~\ref{eq:migration_model_unconditional_symmetric} until frequencies in consecutive generations differ by less than $10^{-4}$, starting with equal frequencies.
}
\label{fig:migration_m}
\end{figure*}

% Discussion
\section*{Discussion}

Most models of cultural transmission assume a fixed relative rate at which different modes of transmission -- vertical, horizontal, or oblique -- occur.
Here we explored a model in which the relative rate of vertical and oblique transmission fluctuates over time, either periodically or randomly, or over space.
Comparing our results with previous results from a similar model with a fixed rate~\citep{Ram2018}, we find that $\ldots$ (to be continued) % TODO
%Specifically, if fluctuations are not small then a single phenotype is likely to fix in the population~(\autoref{eq:poly_condition_periodic_fluc}, \autoref{fig:rho1_rho2_k=1}, \autoref{fig:rho1_rho2_stoch_p}), and 

One caveat of our model is the use of the ``phenotypic gambit'' -- the assumption that the transmission mode itself is strictly vertically transmitted.
Although there is evidence that the tendency to use different learning mechanisms is genetically transmitted~\citep{Foucaud2013}, this assumption can be challenged: individuals may be able to learn how and when to learn, in what has been called ``social learning of social learning''~\citep{Mesoudi2016}.
Indeed, it has been demonstrated that guppy fish are more likely to learn from others if previous social experiences provided benefits~\citep{Leris2016}.
It is also possible that the transmission mode is regulated.
For example, \citet{Farine2015} found that zebra fish switch from vertical to oblique transmission after exposure to stress hormones.
Our model accounts for cases in which the entire population changes its transmission mode due to stress, but not for cases in which only some individuals do so.

An extension of our model could incorporate more sophisticated oblique transmission schemes.
For example, conformity -- preference for learning a frequent phenotype -- has been demonstrated in wild monkeys~\citep{VanDeWaal2013} and birds~\citep{Aplin2015}.
We suggest that the specific mode of oblique transmission can also fluctuate over time, so that individuals can, for example, conform to the frequent phenotype under benign conditions, and prefer a rare phenotype under stressful conditions.
Additional work will be required to understand how such fluctuations affect the population dynamics. 

% Facebook rural internet 
% National service - Israel, Singapore
% Immigrtion - draught
% Opening schools, universities
% sending students abroad, bringing experts from abroad
% foreign labor
% refugees
% hunter-gatherers/chimps increase habitat in years with less food production
% religion - dark ages etc.

% Acknowledgements
%{\small
%\section*{Acknowledgements}
%
%This work was supported in part by 
%the Stanford Center for Computational, Evolutionary and Human Genomics, 
%and the Morrison Institute for Population and Resources Studies, Stanford University.
%}

\newpage

\bibliographystyle{agsm}
%\bibliography{/Users/yoavram/Documents/library}
\bibliography{ms}

\end{document}  