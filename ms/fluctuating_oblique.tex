\documentclass[12pt]{extarticle} %twocolumn
\usepackage{geometry}
\geometry{
a4paper,
total={170mm,257mm},
left=20mm,
top=20mm,
headheight=12pt
}

%\usepackage[parfill]{parskip} % Activate to begin paragraphs with an empty line rather than an indent
\usepackage{graphicx} % Use pdf, png, jpg, or eps§ with pdflatex; use eps in DVI mode
% TeX will automatically convert eps --> pdf in pdflatex		

\usepackage{amssymb,amsmath,amsthm}
\usepackage{commath}
\usepackage{longtable}
\usepackage[hyphens]{url}
\usepackage[dvipsnames]{xcolor}
\usepackage[unicode=true,colorlinks=true,urlcolor=CadetBlue,citecolor=black,linkcolor=black]{hyperref}
\PassOptionsToPackage{hyphens}{url} % url is loaded by hyperref
\usepackage[]{authblk}

%SetFonts
% newtxtext+newtxmath
\usepackage{newtxtext} %loads helv for ss, txtt for tt
\usepackage{amsmath}
\usepackage[bigdelims]{newtxmath}
\usepackage[T1]{fontenc}
\usepackage{textcomp}
%SetFonts

% less space before sections 
% \@startsection {NAME}{LEVEL}{INDENT}{BEFORESKIP}{AFTERSKIP}{STYLE} 
%            optional * [ALTHEADING]{HEADING} 
\makeatletter
 \renewcommand\section{\@startsection {section}{1}{\z@}%
     {-2.5ex \@plus -1ex \@minus -.2ex}%
     {1.3ex \@plus.2ex}%
    {\Large\bfseries}}

% Yoav & Lee commands
\newcommand*{\tr}{^\intercal}
\let\vec\mathbf
\newcommand{\matrx}[1]{{\left[ \stackrel{}{#1}\right]}}
\newcommand{\diag}[1]{\mbox{diag}\matrx{#1}}
\newcommand{\goesto}{\rightarrow}
\newcommand{\dspfrac}[2]{\frac{\displaystyle #1}{\displaystyle #2} }
\newtheorem{theorem}{Theorem}
\newtheorem{corollary}{Corollary}
\newtheorem{lemma}{Lemma}
\newtheorem{remark}{Remark}
\newtheorem{result}{Result}
\renewcommand\qedsymbol{} % no square at end of proof
\newcommand{\cl}{\mathbf{L}}
\newcommand{\cj}{\mathbf{J}}
\newcommand{\ci}{I}

% NatBib
\usepackage[round,colon,authoryear]{natbib}

% Title page
\title{Vertical and oblique transmission with fluctuating transmission}

\author[a]{Yoav Ram}
\author[b]{Uri Liberman}
\author[a]{Marcus W. Feldman}
\affil[a]{Department of Biology, Stanford University, Stanford, CA}
\affil[b]{School of Mathematical Sciences, Tel Aviv University, Israel}

\date{\today}

% Document
\begin{document}
\maketitle

% Abstract
%\begin{abstract}
%\end{abstract}

% Introduction
\section*{Introduction}

Cultural transmission comprises of both vertical transmission (from parent to offspring) and non-vertical transmission, namely horizontal (between peers) and oblique (from non-parental adults to offspring). 
\citet[ch.~3]{Cavalli-Sforza1981} first introduced models in which a specific trait is transmitted either vertically or obliquely.
Others models include different forms of cultural transmission that include both vertical and oblique transmission; usually the transmission strategy of each individual is fixed: for example, \citet{McElreath2008} and \cite{Aoki2005} focused on competitions between individual, vertical (innate), and oblique or horizontal learners; \citet{Fogarty2017} compared the effect of different oblique mechanisms (random, success-biased, best-of-k, one-to-many) on the cultural richness and diversity of the population; and \citet{Aoki2012} modeled scenarios in which individual and social learning occur during separate stages in life. 
In many of these models the effects of fluctuating selection and environmental changes was studied~\citep[reviewed in][]{Aoki2014}. 
For example, we recently studied a model in which each individual can learn his phenotype either from a parent, with probability $\rho$, or from a non-parental adult, with probability $1-\rho$; see \autoref{fig:transmission}.
We found that if selection fluctuates between favoring two phenotypes, but on average favored both phenotypes for similar time periods, then a polymorphism can be maintained~\citep{Ram2018}.

However, the case in which transmission also fluctuates has not received much attention.
Nevertheless, if we assume that social learning can be affected by weather, food availability, population density, and even political atmosphere, then it is only logical that the parameters of transmission will fluctuate over time. 
For example, it may be the case that during cold seasons offspring are more likely to stay close to their parents and therefore learn vertically, whereas during warm seasons they are more likely to wander and interact with other adults, therefore learn obliquely.

Here, we analyze a model in which the probability $\rho$ that individuals learn from their parents (vertical transmission), rather than from non-parental adults (oblique transmission), fluctuates over time.
We find that a polymorphic population cannot be maintained if unless these fluctuations are very small and only weakly correlated with fluctuations in selection.
We suggest that the effect of environmental changes can affect cultural evolution not only through their effect of cultural selection, but also on cultural transmission. 

\begin{figure*}[h]
\centering
\includegraphics[width=0.5\linewidth]{../figures/{transmission}.png}
\caption{
\textbf{Cultural transmission with mixed vertical and oblique transmission.}
When a newborn matures, she will copy her phenotype -- color -- from her mother with probability $\rho$, therefore becoming blue, or from some other female with probability $1-\rho$, in which case her color will depend on the frequency of blue and red adult females.}
\label{fig:transmission}
\end{figure*}

% Model
\section*{Model}

Consider a very large population whose members are characterized by a single dichotomous cultural trait with phenotypes $A$ and $B$ and fitness values $1+s$ and $1$, respectively.
Phenotypes are transmitted vertically with probability $\rho$ or obliquely with probability $1-\rho$.
Given $x$ the frequency of phenotype $A$ at the current generation, the frequency of $A$ in the next generation is
\begin{equation} \label{eq:recurrence}
x' = \rho \frac{1+s}{\overline w} x + (1-\rho)x,
\end{equation}
where $\overline w = 1 + xs$ is the population mean fitness.

% Periodic fluctuating transmission
\subsection*{Periodic fluctuating transmission}

Suppose that the vertical transmission rate $\rho$ is fluctuating, such that $\rho = \rho_1$ in odd generations and $\rho = \rho_2$ in even generations.
From \eqref{eq:recurrence}, the recurrence equations for two generations is then
\begin{equation}\begin{aligned} \label{eq:recurrence_two_generations}
x' = \rho_1 \frac{1+s}{\overline w} x + (1-\rho_1)x, \\
x'' = \rho_2 \frac{1+s}{\overline w} x' + (1-\rho_2)x'.
\end{aligned}\end{equation}

First, fixations of $A$ and $B$ ($x^*=0$ and $x^*=1$, respectively) are both equilibria, as they solve $x''=x$.
Second, because $x'/x = 1+\frac{\rho_1 s (1-x)}{1+sx} > 1$ and $x''/x = 1+\frac{\rho_2 s (1-x)}{1+sx} > 1$, the frequency of $A$ increases every generation and $x^*=1$ is globally stable.
Therefore, we find that fluctuations in the mode of transmission ($\rho$) without fluctuations in selection lead to fixation of the favored phenotype without any stable polymorphisms.

% Periodic fluctuating transmission and selection
\subsection*{Periodic fluctuating transmission and selection}

Now suppose that both the vertical transmission rate $\rho$ and the selection coefficient $s$ fluctuate together.

If the environment changes every $k$ generations, so that for the first $k$ generations $A$ is favored and the transmission rate is $\rho_1$ and for the second $k$ generations $B$ is favored and the transmission rate is $\rho_2$.
The change in the frequency of $A$ when either $A$ or $B$ is favored is described by $F_A(x)$ and $F_B(x)$, respectively, where
\begin{equation}\begin{aligned} \label{eq:recurrence_periodic_fluc}
F_A(x) = \rho_1 \frac{1+s}{1+sx} x + (1-\rho_1)x, \\
F_B(x) = \rho_2 \frac{1}{1+s-sx'} x' + (1-\rho_2)x'.
\end{aligned}\end{equation}
Fixations of $A$ ($x^*=1$) and $B$ ($x^*=0$) are locally stable if, respectively,
\begin{equation}\begin{aligned}
\big(F'_A(1) F'_B(1)\big)^k < 1, \\
\big(F'_A(0) F'_B(0)\big)^k < 1. \\
\end{aligned}\end{equation}
and if these both of these conditions are not met then there exists a protected polymorphism.
Therefore, for a protected polymorphism we require
\begin{equation}
1 < F'_A(1) F'_B(1) = \big(1-\rho_1\frac{s}{1+s}\big)\big(1+\rho_2 s\big) 
= 1-\frac{1}{1+s}\big(\rho_2(1+s-\rho_1)-\rho_1\big),
\end{equation}
and
\begin{equation}
1 < F'_A(0) F'_B(0) = \big(1+\rho_1 s\big)\big(1-\rho_2 \frac{s}{1+s}\big) 
= 1+\frac{s}{1+s}\big(\rho_1(1+s) -\rho_2(1+\rho_1 s)\big),
\end{equation}
which can be summarized as a condition on $\rho_2$
\begin{equation} \label{eq:poly_condition_periodic_fluc}
\frac{\rho_1}{1+s(1-\rho_1)} < \rho_2 < \frac{(1+s)\rho_1}{1+s\rho_1},
\end{equation}
or as a condition on the difference between $\rho_1$ and $\rho_2$
\begin{equation}
-s\rho_2(1-\rho_1) < \rho_2 - \rho_1 < s\rho_1(1-\rho_2).
\end{equation}

\begin{figure*}[hbt]
\centering
\includegraphics[width=0.65\linewidth]{../figures/{rho1_rho2_k=1}.pdf}
\caption{
\textbf{Protected polymorphism with fluctuating transmission and selection.}
The positive root $x^*$ of $G(x)$, see eq.~\ref{eq:xstar_periodic_fluc_k=1}.
Dashed lines represent, from left to right, $\rho_2=\frac{\rho_1}{1+s(1-\rho_1)}$, $\rho_2=\rho_1$, and $\rho_2=\frac{(1+s)\rho_1}{1+s\rho_1}$, see eq.~\ref{eq:poly_condition_periodic_fluc}.
Here, $\rho_1=0.5$, $s=0.1$.}
\label{fig:rho1_rho2_k=1}
\end{figure*}

\autoref{fig:rho1_rho2_k=1} shows the solution for $F_B(F_A(x))=x$ for $k=1$, highlighting the area of the parameter space in which a protected polymorphism exists.
Surprisingly, the stronger selection is, the larger the difference can be that still allows a stable polymorphism.

In the case of $k=1$ we can also find the frequency of $A$ at the protected polymorphism, $x^*$.
Denote $F_B(F_A(x))-x = G(x) \cdot x (1-x) \cdot \gamma$ with
\begin{align} \label{eq:xstar_periodic_fluc_k=1}
G(x) = ax^2+bx+c, \\
\gamma = -\frac{1+xs}{s} \Big( 1 + s + s^2x (1-\rho) +s^2x^2(1-\rho) \Big) > 0, \\
a = s^2 \rho_1 (1-\rho_1) (1 - \rho_2 + \rho_2/\rho_1)) > 0, \\
b = s(1-\rho_1)(2\rho_2 - s\rho_1(1-\rho_2)), \\
c = \rho_2 - \rho_1 - s\rho_1(1-\rho_2),
\end{align} 
and then $x^*$ is a solution of the quadratic $G(x)=0$, given by $x^* = \frac{-b \pm \sqrt{b^2-4ac}}{2a}$.
The condition \eqref{eq:poly_condition_periodic_fluc} guarantees that $x^* \in (0,1)$ and is equivalent to $c<0$.% and $a+b+c>0$.
Therefore, $\sqrt{b^2-4ac} > b$, and if $0<s<1$ (such that $b$ is guaranteed to be positive) then $x^*= \frac{-b+\sqrt{b^2-4ac}}{2a}$ (\autoref{fig:rho1_rho2_k=1}).


 
%\begin{align}
%G(0) = c > 0  \Leftrightarrow \rho_2 > \frac{\rho_1(1+s)}{1+\rho_1s}, \\
%G'(0) = b > 0 \Leftrightarrow \rho_2 > \frac{\rho_1 s}{2 + \rho_1s}, \\
%G(1) = a + b + c = (1+s)[\rho_2 - \rho_1 + s \rho_2 (1-\rho_1)], \\
%G'(1) = 2a + b = s(1-\rho_1)(2(1+s)\rho_2 + s\rho_1(1-\rho_2)) > 0.
%\end{align} 

% Randomly fluctuating transmission and selection
\subsection*{Randomly fluctuating transmission and selection}

We now consider the case that both transmission and selection fluctuate randomly.
Rewrite eq.~28 from \citet{Ram2018} so that $\rho$ is also a random variable
\begin{equation}
x_t = x_t \frac{1 + \rho_t s_t + x_t (1 - \rho_t) s_t}{1 + x_t s_t},
\end{equation}
where $s_t$ are independent and identically distributed and also $\rho_t$ are independent and identically distributed ($t=0,1,2,\ldots$), $P(-1+C<s_t<D)=1$ for some positive $C$ and $D$, and $0<\rho_t<1$.
Therefore, $z_t = \rho_t s_t$  are independent and identically distributed and $P(-1+C < z_t < D$ and from Result~6 and 7 of \citet{Ram2018} we have:
\begin{itemize}
\item Suppose $E[log(1+\rho_t s_t)]>0$. Then $x^*=0$ is not stochastically locally stable and in fact $P(lim_{t \to \infty} x_t=0) = 0$, i.e., fixation of $B$ almost surely does not occur.
\item Suppose $E[log(1+\rho_t s_t)]<0$. Then $x^*=0$ is stochastically locally stable and 
\item Similarly, $E[log(1-\rho_t s_t/(1+s_t)]<0$ leads to stochastic local stability of $x^*=1$, and $E[log(1-\rho_t s_t/(1+s_t))]>0$ assures that fixation of $A$ almost surely does not occur.
\item In particular, if $E[\rho_t s_t] = cov(\rho_t, s_t) + E[\rho_t] E[s_t] \le 0$ then $x^*=0$ is stochastically locally stable, and similarly if $E[-\rho_t s_t/(1+s_t)] = cov(\rho_t, -s_t/(1+s_t)) - E[\rho_t] E[s_t/(1+s_t)] \le 0$ then $x^*=1$ is stochastically locally stable.
\item Note that it is not possible that both $E[log(1+\rho_t s_t)]$ and $E[log(1-\rho_t s_t/(1+s_t)]$ are negative, as their sum is positive
\begin{multline}
E[log(1+\rho_t s_t)] + E[log(1-\rho_t s_t/(1+s_t)] = \\
E[log(1+\rho_t s_t) + log(1-\rho_t s_t/(1+s_t)]= \\
E[log\big((1+\rho_t s_t)(1-\rho_t s_t/(1+s_t)\big)]= \\
E[log\big( 1+\rho_t(1-\rho_t)(s_t^2)/(1+s_t) \big)] > 0,
\end{multline}
and therefore it is not possible that both fixations are stochastically locally stable.
\end{itemize}

We will now present a number of examples.
First, if $s_t$ and $\rho_t$ are independent ($cov(s_t, \rho_t)$) and $s_t$ is symmetric around zero, then $E[\rho_t s_t]=0$ and $E[-\rho_t s_t/(1+s_t)]>0$ (because $E[s_t/(1+s_t)] < E[s_t]$).
Therefore, fixation of $B$ is stochastically locally stable and fixation of $A$ almost surely doesn't occur.
For example, let $s_t \sim U(-1, 1)$ and $\rho_t \sim U(0,1)$ independently (in particular, $cov(\rho_t, s_t)=0$), then $E[log(1+\rho_t s_t)]\approx -0.07315$ and $E[log(1-\rho_t s_t/(1+s_t))]\approx 0.2337$.
However, note that symmetry of $s_t$ around zero provides an advantage to phenotype $B$ -- using Jensen's inequality, $E[w_A/w_B] = E[1+s_t] = 1 \le E[1/(1+s_t)] = E[w_B/w_A]$.
Therefore, if we take $w_{A,t}, w_{B,t} \sim U(0,1)$ independently, and $s_t=(w_A-w_B)/w_B$, then neither $A$ nor $B$ has an advantage, on average (i.e. $E[w_A/w_B]=E[w_B/w_A]$, see \autoref{fig:beta}A), and both $E[\rho_t s_t]$ and $E[-\rho_t s_t/(1+s_t)]$ are positive, so that both fixations are not stochastically locally stable.

If instead we do not assume independence of $s_t$ and $\rho_t$, we can get fixation. Let $\rho_t \sim \beta(1+s, 1)$, where $s_t=(w_A-w_B)/w_B$ and $w_A, w_B \sim U(0,1)$, such that the covariance of $s_t$ and $\rho_t$ is positive ($cov(s_t, \rho_t) \approx 4$), that is, vertical transmission is more likely when $A$ if favored (when $s_t>0$) and oblique transmission is more  likely when $B$ is favored (when $s_t<0$; \autoref{fig:beta}).
Then $E[log(1+\rho_t s_t)] >0$ and $B$ almost surely doesn't fix, and $E[log(1-\rho_t s_t / (1+s_t)] < 0$, so fixation of $A$ is stochastically locally stable.
The opposite occurs if $\rho_t \sim \beta(1, 1+s)$ and the covariance of $s_t$ and $\rho_t$ is negative ($cov(s_t, \rho_t) \approx -4$). In that case, fixation of $B$ is stochastically locally stable and $A$ almost surely doesn't fix.

\begin{figure*}[hbt]
\centering
\includegraphics[width=\linewidth]{../figures/{beta}.png}
\caption{
\textbf{Covariance of selection and transmission.}
\textbf{(A)} Histogram of $w_A/w_B$ where $w_A$ and $w_B$ are identically and independently distributed uniform random variables $U(0,1)$.
\textbf{(B)} Histogram of $s_t = (w_A-w_B)/w_B$.
\textbf{(C)} Histogram of $\rho_t \sim Beta(1+s_t, 1)$.
\textbf{(C)} The joint distribution of $\rho_t$ and $s_t$ demonstrates a positive correlation $cov(s_t, \rho_t)>0$.
}
\label{fig:beta}
\end{figure*}

Third, it is also possible that both fixations are not stochastically locally stable even if $s_t$ and $\rho_t$ covary, but, similar to the case of periodic fluctuations, this can only occur if the fluctuations in $\rho_t$ are small.
For example, \autoref{fig:rho1_rho2_stoch_p} shows the expected outcome when $s_t=s$ and $\rho_t=\rho_1$ with probability $p=0.505$ and $s_t=-s$ and $\rho_t=\rho_2$ with probability $1-p=0.495$.
The blue and red areas denote expected fixation of $A$ and $B$, respectively (i.e. stochastic local stability) and the white area shows expected polymorphism (both fixations are not stochastically locally stable). 

\begin{figure*}[hbt]
\centering
\includegraphics[width=0.75\linewidth]{../figures/{rho1_rho2_stoch_p}.pdf}
\caption{
\textbf{Stochastic local stability.}
Here, $s_t=s$ and $\rho_t=\rho_1$ with probability $p=0.505$ and $s_t=-s$ and $\rho_t=\rho_2$ with probability $1-p=0.495$, with $s=0.05$ and $p=0.505$, a value combination that allowed a stochastic polymorphism with a fixed vertical transmission rate $\rho=0.1$ in~\citet[Fig.~2]{Ram2018}.}
\label{fig:rho1_rho2_stoch_p}
\end{figure*}

% Finite population size
\subsection*{Finite population size}

To include the effect of random drift due to finite population size in the above deterministic model, we follow \citet{Ram2018} and develop a diffusion equation approximation.
In~\citet{Ram2018} only selection fluctuated via $s_t$, but here we also transmission to fluctuate via $\rho_t$. 

We find a result similar to Result 11 from~\citet{Ram2018}:
The mean $\mu(x)$ and variance $\sigma^2(x)$ of the change in the frequency of $A$ in one generation for the diffusion approximation in the case of a cycling environment $AkBl$, where $k+l=n$, are
\begin{equation}
\mu(x) = S_n x(1-x), \quad \text{and} \quad \sigma^2(x) = n x (1-x)
\end{equation}
where $S_n = \sum_{t=1}^{n}{\rho_t s_t}$, which is a weighted average of the selection coefficients in favor of phenotype $A$, weighted by the vertical transmission rate.

% Results
%\section*{Results}

% Discussion
%\section*{Discussion}

% Acknowledgements
%{\small
%\section*{Acknowledgements}
%
%This work was supported in part by 
%the Stanford Center for Computational, Evolutionary and Human Genomics, 
%and the Morrison Institute for Population and Resources Studies, Stanford University.
%}

\bibliographystyle{agsm}
\bibliography{/Users/yoavram/Documents/library}
%\bibliography{fluctuating_oblique}

\end{document}  