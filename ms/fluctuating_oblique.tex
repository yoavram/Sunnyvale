\documentclass[12pt]{extarticle} %twocolumn
\usepackage{geometry}
\geometry{
a4paper,
total={170mm,257mm},
left=20mm,
top=20mm,
headheight=12pt
}

%\usepackage[parfill]{parskip} % Activate to begin paragraphs with an empty line rather than an indent
\usepackage{graphicx} % Use pdf, png, jpg, or eps§ with pdflatex; use eps in DVI mode
% TeX will automatically convert eps --> pdf in pdflatex		

\usepackage{amssymb,amsmath,amsthm}
\usepackage{commath}
\usepackage{longtable}
\usepackage[hyphens]{url}
\usepackage[dvipsnames]{xcolor}
\usepackage[unicode=true,colorlinks=true,urlcolor=CadetBlue,citecolor=black,linkcolor=black]{hyperref}
\PassOptionsToPackage{hyphens}{url} % url is loaded by hyperref
\usepackage[]{authblk}

%SetFonts
% newtxtext+newtxmath
\usepackage{newtxtext} %loads helv for ss, txtt for tt
\usepackage{amsmath}
\usepackage[bigdelims]{newtxmath}
\usepackage[T1]{fontenc}
\usepackage{textcomp}
%SetFonts

% less space before sections 
% \@startsection {NAME}{LEVEL}{INDENT}{BEFORESKIP}{AFTERSKIP}{STYLE} 
%            optional * [ALTHEADING]{HEADING} 
\makeatletter
 \renewcommand\section{\@startsection {section}{1}{\z@}%
     {-2.5ex \@plus -1ex \@minus -.2ex}%
     {1.3ex \@plus.2ex}%
    {\Large\bfseries}}

% Yoav & Lee commands
\newcommand*{\tr}{^\intercal}
\let\vec\mathbf
\newcommand{\matrx}[1]{{\left[ \stackrel{}{#1}\right]}}
\newcommand{\diag}[1]{\mbox{diag}\matrx{#1}}
\newcommand{\goesto}{\rightarrow}
\newcommand{\dspfrac}[2]{\frac{\displaystyle #1}{\displaystyle #2} }
\newtheorem{theorem}{Theorem}
\newtheorem{corollary}{Corollary}
\newtheorem{lemma}{Lemma}
\newtheorem{remark}{Remark}
\newtheorem{result}{Result}
\renewcommand\qedsymbol{} % no square at end of proof
\newcommand{\cl}{\mathbf{L}}
\newcommand{\cj}{\mathbf{J}}
\newcommand{\ci}{I}

% NatBib
\usepackage[round,colon,authoryear]{natbib}

% Title page
\title{Vertical and oblique transmission with fluctuating transmission}

\author[a]{Yoav Ram}
\author[b]{Uri Liberman}
\author[a]{Marcus W. Feldman}
\affil[a]{Department of Biology, Stanford University, Stanford, CA}
\affil[b]{School of Mathematical Sciences, Tel Aviv University, Israel}

\date{\today}

% Document
\begin{document}
\maketitle

% Abstract
%\begin{abstract}
%\end{abstract}

% Introduction
%\section*{Introduction}


% Model
\section*{Model}

Consider a very large population whose members are characterized by a single dichotomous cultural trait with phenotypes $A$ and $B$ and fitness values $1+s$ and $1$, respectively.
Phenotypes are transmitted vertically with probability $\rho$ or obliquely with probability $1-\rho$.
Given $x$ the frequency of phenotype $A$ at the current generation, the frequency of $A$ in the next generation is
\begin{equation} \label{eq:recurrence}
x' = \rho \frac{1+s}{\overline w} x + (1-\rho)x,
\end{equation}
where $\overline w = 1 + xs$ is the population mean fitness.

% Periodic fluctuating transmission
\subsection*{Periodic fluctuating transmission}

Suppose that the vertical transmission rate $\rho$ is fluctuating, such that $\rho = \rho_1$ in odd generations and $\rho = \rho_2$ in even generations.
From \eqref{eq:recurrence}, the recurrence equations for two generations is then
\begin{equation}\begin{aligned} \label{eq:recurrence_two_generations}
x' = \rho_1 \frac{1+s}{\overline w} x + (1-\rho_1)x, \\
x'' = \rho_2 \frac{1+s}{\overline w} x' + (1-\rho_2)x'.
\end{aligned}\end{equation}

First, fixations of $A$ and $B$ ($x^*=0$ and $x^*=1$, respectively) are both equilibria, as they solve $x''=x$.
Second, because $x'/x = 1+\frac{\rho_1 s (1-x)}{1+sx} > 1$ and $x''/x = 1+\frac{\rho_2 s (1-x)}{1+sx} > 1$, the frequency of $A$ increases every generation and $x^*=1$ is globally stable.
Therefore, we find that fluctuations in the mode of transmission ($\rho$) without fluctuations in selection lead to fixation of the favored phenotype without any stable polymorphisms.

% Periodic fluctuating transmission and selection
\subsection*{Periodic fluctuating transmission and selection}

Now suppose that both the vertical transmission rate $\rho$ and the selection coefficient $s$ fluctuate together.

If the environment changes every $k$ generations, so that for the first $k$ generations $A$ is favored and the transmission rate is $\rho_1$ and for the second $k$ generations $B$ is favored and the transmission rate is $\rho_2$.
The change in the frequency of $A$ when either $A$ or $B$ is favored is described by $F_A(x)$ and $F_B(x)$, respectively, where
\begin{equation}\begin{aligned} \label{eq:recurrence_periodic_fluc}
F_A(x) = \rho_1 \frac{1+s}{1+sx} x + (1-\rho_1)x, \\
F_B(x) = \rho_2 \frac{1}{1+s-sx'} x' + (1-\rho_2)x'.
\end{aligned}\end{equation}
Fixations of $A$ ($x^*=1$) and $B$ ($x^*=0$) are locally stable if, respectively,
\begin{equation}\begin{aligned}
\big(F'_A(1) F'_B(1)\big)^k < 1, \\
\big(F'_A(0) F'_B(0)\big)^k < 1. \\
\end{aligned}\end{equation}
and if these both of these conditions are not met then there exists a protected polymorphism.
Therefore, for a protected polymorphism we require
\begin{equation}
1 < F'_A(1) F'_B(1) = \big(1-\rho_1\frac{s}{1+s}\big)\big(1+\rho_2 s\big) 
= 1-\frac{1}{1+s}\big(\rho_2(1+s-\rho_1)-\rho_1\big),
\end{equation}
and
\begin{equation}
1 < F'_A(0) F'_B(0) = \big(1+\rho_1 s\big)\big(1-\rho_2 \frac{s}{1+s}\big) 
= 1+\frac{s}{1+s}\big(\rho_1(1+s) -\rho_2(1+\rho_1 s)\big),
\end{equation}
which can be summarized as a condition on $\rho_2$
\begin{equation} \label{eq:poly_condition_periodic_fluc}
\frac{\rho_1}{1+s(1-\rho_1)} < \rho_2 < \frac{(1+s)\rho_1}{1+s\rho_1},
\end{equation}
or as a condition on the difference between $\rho_1$ and $\rho_2$
\begin{equation}
-s\rho_2(1-\rho_1) < \rho_2 - \rho_1 < s\rho_1(1-\rho_2).
\end{equation}

\begin{figure*}[hbt]
\centering
\includegraphics[width=\linewidth]{../figures/{rho1_rho2_k=1}.pdf}
\caption{
\textbf{Protected polymorphism with fluctuating transmission and selection.}
The positive root $x^*$ of $G(x)$, see eq.~\ref{eq:xstar_periodic_fluc_k=1}.
Dashed lines represent, from left to right, $\rho_2=\frac{\rho_1}{1+s(1-\rho_1)}$, $\rho_2=\rho_1$, and $\rho_2=\frac{(1+s)\rho_1}{1+s\rho_1}$, see eq.~\ref{eq:poly_condition_periodic_fluc}.
Here, $\rho_1=0.5$, $s=0.1$.}
\label{fig:rho1_rho2_k=1}
\end{figure*}

\autoref{fig:rho1_rho2_k=1} shows the solution for $F_B(F_A(x))=x$ for $k=1$, highlighting the area of the parameter space in which a protected polymorphism exists.
Surprisingly, the stronger selection is, the larger the difference can be that still allows a stable polymorphism.

In the case of $k=1$ we can also find the frequency of $A$ at the protected polymorphism, $x^*$.
Denote $F_B(F_A(x))-x = G(x) \cdot x (1-x) \cdot \gamma$ with
\begin{align} \label{eq:xstar_periodic_fluc_k=1}
G(x) = ax^2+bx+c, \\
\gamma = -\frac{1+xs}{s} \Big( 1 + s + s^2x (1-\rho) +s^2x^2(1-\rho) \Big) > 0, \\
a = s^2 \rho_1 (1-\rho_1) (1 - \rho_2 + \rho_2/\rho_1)) > 0, \\
b = s(1-\rho_1)(2\rho_2 - s\rho_1(1-\rho_2)), \\
c = \rho_2 - \rho_1 - s\rho_1(1-\rho_2),
\end{align} 
and then $x^*$ is a solution of the quadratic $G(x)=0$, given by $x^* = \frac{-b \pm \sqrt{b^2-4ac}}{2a}$.
The conditions for existence of the protected polymorphism \eqref{eq:poly_condition_periodic_fluc} is equivalent to $c<0$.% and $a+b+c>0$.
Therefore, $\sqrt{b^2-4ac} > b$, and if $0<s<1$ (such that $b$ is guaranteed to be positive) then $x^*= \frac{-b+\sqrt{b^2-4ac}}{2a}$ (\autoref{fig:rho1_rho2_k=1}).


 
%\begin{align}
%G(0) = c > 0  \Leftrightarrow \rho_2 > \frac{\rho_1(1+s)}{1+\rho_1s}, \\
%G'(0) = b > 0 \Leftrightarrow \rho_2 > \frac{\rho_1 s}{2 + \rho_1s}, \\
%G(1) = a + b + c = (1+s)[\rho_2 - \rho_1 + s \rho_2 (1-\rho_1)], \\
%G'(1) = 2a + b = s(1-\rho_1)(2(1+s)\rho_2 + s\rho_1(1-\rho_2)) > 0.
%\end{align} 

% Randomly fluctuating transmission and selection
\subsection*{Randomly fluctuating transmission and selection}

We now consider the case that both transmission and selection fluctuate randomly.
Rewrite eq.~28 from \citet{Ram2018} so that $\rho$ is also a random variable
\begin{equation}
x_t = x_t \frac{1 + \rho_t s_t + x_t (1 - \rho_t) s_t}{1 + x_t s_t},
\end{equation}
where $s_t$ are independent and identically distributed and also $\rho_t$ are independent and identically distributed ($t=0,1,2,\ldots$), $P(-1+C<s_t<D)=1$ for some positive $C$ and $D$, and $0<\rho_t<1$.
Therefore, $z_t = \rho_t s_t$  are independent and identically distributed and $P(-1+C < z_t < D$ and from Result~6 and 7 of \citet{Ram2018} we have:
\begin{itemize}
\item Suppose $E[log(1+\rho_t s_t)]>0$. Then $x^*=0$ is not stochastically locally stable and in fact $P(lim_{t \to \infty} x_t=0) = 0$, i.e., fixation of $B$ almost surely does not occur.
\item Suppose $E[log(1+\rho_t s_t)]<0$. Then $x^*=0$ is stochastically locally stable and 
\item Similarly, $E[log(1-\rho_t s_t/(1+s_t)]<0$ leads to stochastic local stability of $x^*=1$, and $E[log(1-\rho_t s_t/(1+s_t))]>0$ assures that fixation of $A$ almost surely does not occur.
\item In particular, if $E[\rho_t s_t] = cov(\rho_t, s_t) + E[\rho_t] E[s_t] \le 0$ then $x^*=0$ is stochastically locally stable, and similarly if $E[-\rho_t s_t/(1+s_t)] = cov(\rho_t, -s_t/(1+s_t)) - E[\rho_t] E[s_t/(1+s_t)] \le 0$ then $x^*=1$ is stochastically locally stable.
\end{itemize}

We will now present a number of examples.
First, if $s_t \sim U(-1, 1)$ and $\rho_t \sim U(0,1)$ independently (in particular, $cov(\rho_t, s_t)=0$), then both $E[log(1+\rho_t s_t)]$ and $E[log(1-\rho_t s_t/(1+s_t))]$ are negative, and therefore both fixations ($A$ and $B$) are stochastically locally stable. 
It is important to note that without fluctuations in $\rho$ (i.e. fixed $\rho$) it is not possible to have both fixations stable~\citep{Ram2018}.

Second, if $s_t$ is symmetric around zero ($E[s_t]=E[s_t/(1+s_t)]=0$), and $\rho_t$ and $s_t$ are positively (negatively) correlated, such that vertical transmission is more likely when $A$ ($B$) is favored, then $E[-\rho_t s_t/(1+s_t)]<0$ ($E[-\rho_t s_t]<0$) and therefore fixation of $A$ ($B$) is stochastically locally stable.
For example, let $s_t \sim U(-1,1)$ and $\rho_t \sim \beta(1+s, 1)$, such that the covariance of $s_t$ and $\rho_t$ is positive ($cov(s_t, \rho_t) \approx 0.22$), that is, vertical transmission is more likely when $A$ if favored (when $s_t>0$) and oblique transmission is more  likely when $B$ is favored (when $s_t<0$), then ($Beta$ being the beta function), 
\begin{equation}
E[log(1+\rho_t s_t)] = \frac{1}{2} \int_{-1}^{1} {\int_0^1 {\frac{\rho^s \log{(1+\rho s)}}{Beta(1+s, 1)}  \; d\rho}\; ds} \approx 0.2525,
\end{equation}
and $B$ almost surely doesn't fix, and
\begin{equation}
E[log(1-\rho_t s_t / (1+s_t)] = \frac{1}{2} \int_{-1}^{1} {\int_0^1 {\frac{(1-\rho)^{-s/(1+s)} \log{(1-\rho s / (1+s))}}{Beta(1, 1-s/(1+s)}  \; d\rho}\; ds} \approx -0.076,
\end{equation}
so fixation of $A$ is stochastically locally stable.
The opposite occurs if $\rho_t \sim \beta(1+s, 1)$ and therefore the covariance of $s_t$ and $\rho_t$ is negative ($cov(s_t, \rho_t) \approx -0.2$); in that case, fixation of $B$ is stochastically locally stable and $A$ almost surely doesn't fix.

Third, it is also possible that both fixations are stochastically locally unstable, but, similar to the case of periodic fluctuations, this can only occur if the fluctuations in $\rho_t$ are small.
For example, \autoref{fig:rho1_rho2_stoch_p} shows the expected outcome when $s_t=s$ and $\rho_t=\rho_1$ with probability $p=0.505$ and $s_t=-s$ and $\rho_t=\rho_2$ with probability $1-p=0.495$. The blue and red areas denote expected fixation of $A$ and $B$, respectively (i.e. stochastic local stability) and the white area shows expected polymorphism (both fixations are not stochastically locally stable). 

\begin{figure*}[hbt]
\centering
\includegraphics[width=0.75\linewidth]{../figures/{rho1_rho2_stoch_p}.pdf}
\caption{
\textbf{Stochastic local stability.}
Here, $s_t=s$ and $\rho_t=\rho_1$ with probability $p=0.505$ and $s_t=-s$ and $\rho_t=\rho_2$ with probability $1-p=0.495$, with $s=0.05$ and $p=0.505$, a value combination that allowed a stochastic polymorphism with a fixed vertical transmission rate $\rho=0.1$ in~\citet[Fig.~2]{Ram2018}.}
\label{fig:rho1_rho2_stoch_p}
\end{figure*}

% Finite population size
\subsection*{Finite population size}

To include the effect of random drift due to finite population size in the above deterministic model, we follow \citet{Ram2018} and develop a diffusion equation approximation.
In~\citet{Ram2018} only selection fluctuated via $s_t$, but here we also transmission to fluctuate via $\rho_t$. 

We find a result similar to Result 11 from~\citet{Ram2018}:
The mean $\mu(x)$ and variance $\sigma^2(x)$ of the change in the frequency of $A$ in one generation for the diffusion approximation in the case of a cycling environment $AkBl$, where $k+l=n$, are
\begin{equation}
\mu(x) = S_n x(1-x), \quad \text{and} \quad \sigma^2(x) = n x (1-x)
\end{equation}
where $S_n = \sum_{t=1}^{n}{\rho_t s_t}$, which is a weighted average of the selection coefficients in favor of phenotype $A$, weighted by the vertical transmission rate.

% Results
%\section*{Results}

% Discussion
%\section*{Discussion}

% Acknowledgements
%{\small
%\section*{Acknowledgements}
%
%This work was supported in part by 
%the Stanford Center for Computational, Evolutionary and Human Genomics, 
%and the Morrison Institute for Population and Resources Studies, Stanford University.
%}

\bibliographystyle{agsm}
%\bibliography{/Users/yoavram/Documents/library}
\bibliography{fluctuating_oblique}

\end{document}  