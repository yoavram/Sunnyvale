\documentclass[14pt]{extarticle}
\usepackage{geometry}
\geometry{
a4paper,
total={170mm,252mm},
left=20mm,
top=20mm,
headheight=12pt
}

\usepackage[parfill]{parskip} % Activate to begin paragraphs with an empty line rather than an indent
\usepackage{graphicx} % Use pdf, png, jpg, or eps§ with pdflatex; use eps in DVI mode
% TeX will automatically convert eps --> pdf in pdflatex		

\usepackage{amssymb,amsmath,amsthm}
\usepackage{commath}
\usepackage{longtable}
\usepackage[hyphens]{url}
\usepackage[dvipsnames]{xcolor}
\usepackage[unicode=true,colorlinks=true,urlcolor=CadetBlue,citecolor=black,linkcolor=black]{hyperref}
\PassOptionsToPackage{hyphens}{url} % url is loaded by hyperref
\usepackage[]{authblk}

%SetFonts
% newtxtext+newtxmath
\usepackage{newtxtext} %loads helv for ss, txtt for tt
\usepackage{amsmath}
\usepackage[bigdelims]{newtxmath}
\usepackage[T1]{fontenc}
\usepackage{textcomp}
%SetFonts

% less space before sections 
% \@startsection {NAME}{LEVEL}{INDENT}{BEFORESKIP}{AFTERSKIP}{STYLE} 
%            optional * [ALTHEADING]{HEADING} 
\makeatletter
 \renewcommand\section{\@startsection {section}{1}{\z@}%
     {-2.5ex \@plus -1ex \@minus -.2ex}%
     {1.3ex \@plus.2ex}%
    {\Large\bfseries}}

% Yoav & Lee commands
\newcommand*{\tr}{^\intercal}
\let\vec\mathbf
\newcommand{\matrx}[1]{{\left[ \stackrel{}{#1}\right]}}
\newcommand{\diag}[1]{\mbox{diag}\matrx{#1}}
\newcommand{\goesto}{\rightarrow}
\newcommand{\dspfrac}[2]{\frac{\displaystyle #1}{\displaystyle #2} }
\newtheorem{theorem}{Theorem}
\newtheorem{corollary}{Corollary}
\newtheorem{lemma}{Lemma}
\newtheorem{remark}{Remark}
\newtheorem{result}{Result}
\renewcommand\qedsymbol{} % no square at end of proof
\newcommand{\cl}{\mathbf{L}}
\newcommand{\cj}{\mathbf{J}}
\newcommand{\ci}{I}

% NatBib
\usepackage[round,colon,authoryear]{natbib}
\usepackage{xcolor}

% line numbers
\usepackage[displaymath, mathlines]{lineno}
\renewcommand\linenumberfont{\normalfont\small\sffamily}
\linenumbers
\modulolinenumbers[2]

% Title page
\title{Vertical and oblique cultural transmission fluctuating in time and in space}

\author[a]{Yoav Ram}
\author[b]{Uri Liberman}
\author[a]{Marcus W. Feldman}
\affil[a]{Department of Biology, Stanford University, Stanford, CA}
\affil[b]{School of Mathematical Sciences, Tel Aviv University, Israel}

\date{\today}

% Document
\begin{document}
\maketitle

% Abstract
%\begin{abstract}
%\end{abstract}

% Introduction
\section*{Introduction}

In evolutionary genetics, the properties of transmission are known to be important, e.g., uniparental vs.\ biparental; haploid vs.\ diploid; with or without recombination; sexual or asexual. For cultural evolution, the mode of transmission between individuals of the same or different generations is also central to the dynamics of cultural traits.
Using analysis with epidemic theory, \citet[ch.~3]{Cavalli-Sforza1981}  introduced models in which a specific cultural trait is transmitted either vertically, that is, directly from parent to offspring, obliquely, that is, from non-parental members of the parent's generation to an offspring, or horizontally, that is, among members of the same generational cohort.
Mathematical models of cultural evolution may include one or more modes of cultural transmission and in most of these models the mode of transmission by individuals is fixed; that is, it does not vary over time {\color{red}or space}.
For example, \cite{Feldman1996}, \cite{Wakano2004}, \cite{Aoki2005}, and \citet{McElreath2008} focused on competition between individual learning, innate learning, and social learning; \citet{Fogarty2017} compared the effects of different oblique mechanisms (random, success-biased, best-of-$k$, one-to-many) on the cultural richness and diversity of the population; and \citet{Aoki2012} modeled scenarios in which individual and social learning occur during separate stages in life.
In these studies, the cultural transmission rule did not change during the evolutionary process.

Several of these models included fluctuating selection, which could be due to environmental changes~\citep[reviewed in][]{Aoki2014}. 
We recently studied a model in which each individual can learn a dichotomous phenotype either from a parent, with probability $\rho$, or from an adult in the parental generation, with probability $1-\rho$ (\autoref{fig:transmission}).
We found that if selection fluctuates between favoring each of the two phenotypes, but on average favored both phenotypes for similar time periods, then a phenotypic polymorphism may be maintained~\citep{Ram2018}.
Furthermore, we found that if the environment changes very rapidly then lower $\rho$ values are likely to evolve, that is, oblique transmission is favored over vertical transmission.

There has been much less theory developed for fluctuating transmission.
Nevertheless, if we assume that social learning can be affected by ecological and demographic factors, such as the frequency of interactions~\citep{VanSchaik2003}, weather~\citep{Phithakkitnukoon2012}, population size or density~\citep{Fischer2015,Aureli1997a}, or stress~\citep{Farine2015}, then it is reasonable that the mode and rate of transmission may fluctuate over time and/or space.
For example, \citet{Webster2008} have demonstrated that minnows (small freshwater shoaling fish), increase their reliance on social learning when predation risk increases. 

Here, we analyze a model~(\autoref{fig:transmission}) in which individuals learn from their parents with probability $\rho$ (vertical transmission), and from adults in the parental generation with probability $1-\rho$ (oblique transmission), where $\rho$ fluctuates over time or space.
We find that fluctuations in $\rho$ without corresponding fluctuations in selection cannot maintain a polymorphism.
When both the rate of vertical transmission and selection fluctuate together in a deterministic setting, the stronger the selection, the greater is the difference in vertical transmission rates that maintains a polymorphism.
We also study a two-population model with migration, symmetrically different fitnesses in the two populations, and different rates of vertical transmission in the two populations.
Stability of fixation points  and a polymorphism are shown to depend on strength of selection as well as rates of migration and transmission.

We suggest that cultural evolution can be affected by environmental changes that cause temporal or spatial variation in either selection or transmission or both, and that further empirical and theoretical studies are required to disentangle these effects.

% figure
\begin{figure*}[h]
\centering
\includegraphics[width=0.5\linewidth]{../figures/{transmission}.png}
\caption{
\textbf{Cultural transmission with mixed vertical and oblique transmission.}
When a newborn matures, she copies the phenotype---color---of her mother with probability $\rho$, therefore becoming blue, or from some other female with probability $1-\rho$, in which case her color will depend on the frequency of blue and red adult females.}
\label{fig:transmission}
\end{figure*}

% Models and Results
\section*{Models and Results}

Consider a very large population whose members are characterized by a single dichotomous cultural trait with phenotypes $A$ and $B$ and fitness values $w_A=1+s$ and $w_B=1$, respectively.
Phenotypes are transmitted vertically with probability $\rho$ or obliquely with probability $1-\rho$~(\autoref{fig:transmission}).
Given $x$, the frequency of phenotype $A$ at the current generation, the frequency of $A$ in the next generation is
\begin{equation} \label{eq:recurrence}
x' = \rho \frac{1+s}{\overline w} x + (1-\rho)x,
\end{equation}
where $\overline w = 1 + xs$ is the population mean fitness.

% Periodically fluctuating transmission
\subsection*{Periodically fluctuating transmission}

Suppose that the vertical transmission rate $\rho$ fluctuates, with $\rho = \rho_1$ in odd generations and $\rho = \rho_2$ in even generations.
From \eqref{eq:recurrence}, the recurrence equations for two generations is
\begin{equation}\begin{aligned} \label{eq:recurrence_two_generations}
x' = \rho_1 \frac{1+s}{\overline w} x + (1-\rho_1)x, \quad \overline w = 1 + xs, \\
x'' = \rho_2 \frac{1+s}{\overline w'} x' + (1-\rho_2)x', \quad \overline w' = 1 + x's.
\end{aligned}\end{equation}

First, fixations of $A$ and $B$ ($x^*=1$ and $x^*=0$, respectively) are both equilibria, as they solve $x''=x$.
Second, when $s>0$ ($s<0$) because $x'/x = 1+\frac{\rho_1 s (1-x)}{1+sx} > 1$ ($<1$) and $x''/x' = 1+\frac{\rho_2 s (1-x')}{1+sx'} > 1$ ($<1$), the frequency of $A$ ($B$) increases every generation and $x^*=1$ ($x^*=0$) is globally stable.
Therefore, fluctuations in the mode of transmission ($\rho$) without fluctuations in selection lead to fixation of the favored phenotype, and there cannot be a stable polymorphism.

% Periodically fluctuating transmission and selection
\subsection*{Periodically fluctuating transmission and selection}

Suppose that both transmission and selection fluctuate together, so that when $A$ is favored, with $w_A=1+s$ and $w_B=1$, the transmission rate is $\rho_A$, and when $B$ is favored, with $w_A=1$ and $w_B=1+s$, the transmission rate is $\rho_B$.
The change in the frequency $x$ of phenotype $A$ when either $A$ or $B$ is favored is described by $F_A(x)$ or $F_B(x)$, respectively, where
\begin{equation}\begin{aligned} \label{eq:recurrence_periodic_fluc}
&F_A(x) = \rho_A \frac{1+s}{1+sx} x + (1-\rho_A)x, &\quad \text{and} \\
&F_B(x) = \rho_B \frac{1}{1+s-sx} x + (1-\rho_B)x.
\end{aligned}\end{equation}

% Symmetric periods
\paragraph{Symmetric periods---$AkBk$.}
We first consider environments that fluctuate periodically every $k$ generations between favoring $A$ and $B$.
Using a linear approximation, fixations of $A$ ($x^*=1$) and $B$ ($x^*=0$) are locally stable if, respectively,
\begin{equation}\begin{aligned}
\big[F'_A(1) F'_B(1)\big]^k < 1,  &\quad \text{and}\\
\big[F'_A(0) F'_B(0)\big]^k < 1. \\
\end{aligned}\end{equation}
By definition, a \emph{protected polymorphism} exists if neither fixation is stable~\citep{Prout1968}, that is, if neither of these conditions are met.
Therefore, for a protected polymorphism to exist, we require
\begin{equation}
1 < F'_A(1) F'_B(1) = \Big(1-\rho_A\frac{s}{1+s}\Big)\Big(1+\rho_B s\Big) 
= 1+\frac{s}{1+s}\big[\rho_B(1+s-s\rho_A)-\rho_A\big],
\end{equation}
and
\begin{equation}
1 < F'_A(0) F'_B(0) = \Big(1+\rho_A s\Big)\Big(1-\rho_B \frac{s}{1+s}\Big) 
= 1+\frac{s}{1+s}\big[\rho_A(1+s-s\rho_B)-\rho_B\big],
\end{equation}
which can be summarized as a condition on $\rho_B$,
\begin{equation} \label{eq:poly_condition_periodic_fluc}
\frac{\rho_A}{1+s(1-\rho_A)} < \rho_B < \frac{(1+s)\rho_A}{1+s\rho_A},
\end{equation}
or as a condition on the difference between $\rho_B$ and $\rho_A$
\begin{equation}
-s\rho_B(1-\rho_A) < \rho_B - \rho_A < s\rho_A(1-\rho_B).
\end{equation}
We can state this as Result 1. 

{\bf Result 1.} {\sl Suppose that vertical transmission occurs at rate $\rho_A$ when the fitness of phenotype $A$ is $1+s$ relative to $1$ for phenotype $B$, and at rate $\rho_B$ when the fitness of phenotype $B$ is $1+s$ relative to $1$ for phenotype $A$.
Then the stronger the selection (i.e. the greater the value of $s$), the larger the difference in vertical transmission rates that allows a protected polymorphism.}

% Period of one
\paragraph{Period of one---$A1B1$.}
In the case  $k=1$ we can find $x^*$ the stable frequency of $A$ at the protected polymorphism.
Set $F_B(F_A(x))-x = G(x) \cdot x (1-x) \cdot \gamma$ with
\begin{equation} \begin{aligned} \label{eq:xstar_periodic_fluc_k=1}
&G(x) = ax^2+bx+c, \\
&\gamma = -\frac{1 + sx}{s} \Big[ 1 + s + s^2 x (1-\rho)(1+x)\Big], \\
&a = s^2 \rho_A (1-\rho_A) (1 - \rho_B + \rho_B/\rho_A)) > 0, \\
&b = s(1-\rho_A)(2\rho_B - s\rho_A(1-\rho_B)), \\
& c = \rho_B - \rho_A - s\rho_A(1-\rho_B).
\end{aligned} \end{equation}
Then $x^*$ is a solution of the quadratic $G(x)=0$.
The condition \eqref{eq:poly_condition_periodic_fluc}, which guarantees that $0 < x^* < 1$, is equivalent to $c<0$, and therefore it also guarantees that $\sqrt{b^2-4ac} > b$.
So, if $0<s<1$ (so that $b$ is guaranteed to be positive if LHS of eq.~\ref{eq:poly_condition_periodic_fluc} holds), then 
\begin{equation}
x^*= \frac{-b+\sqrt{b^2-4ac}}{2a}.
\end{equation}
\autoref{fig:rho1_rho2_phase_k=1} shows $x^*$ and highlights the area of the parameter space in which a protected polymorphism exists.
The figure demonstrates that the stronger the selection (x-axis), the greater the fluctuations in $\rho$  (y-axis) that still allow a polymorphic population (area between the dashed lines).

% figure
\begin{figure*}[htb]
\centering
\includegraphics[width=0.65\linewidth]{../figures/{rho1_rho2_phase_k=1}.pdf}
\caption{
\textbf{Protected polymorphism.}
The stable equilibrium of the frequency of phenotype $A$ (eq.~\ref{eq:xstar_periodic_fluc_k=1}) for different selection coefficients ($s$ on x-axis) and size of fluctuations in vertical transmission rates ($\rho_B-\rho_A$ on y-axis) when both selection and transmission fluctuate every generation ($k=1$).
Dashed lines represent $\rho_B=\frac{\rho_A}{1+s(1-\rho_A)}$ and $\rho_B=\frac{(1+s)\rho_A}{1+s\rho_A}$, the limits on $\rho_B-\rho_A$ from inequalities~\eqref{eq:poly_condition_periodic_fluc} that permit a protected polymorphism.
Here, $\rho_A=0.5$.}
\label{fig:rho1_rho2_phase_k=1}
\end{figure*}

% Asymmetric period
\paragraph{Asymmetric period---$AkBl$.}
More generally, phenotype $A$ could be favored for $k$ generations and phenotype $B$ for $l$ generations, with the transmission rate following the same cycle with $\rho=\rho_A$ when $A$ is favored and $\rho=\rho_B$ when $B$ is favored.
The requirements for a protected polymorphism are now
\begin{equation}\begin{aligned}
& 1 < (F'_A(0))^k (F'_B(0))^l = a^k \big(b - s \Delta \rho /(1+s)\big)^l, \quad \text{and} \\
& 1 < (F'_A(1))^k (F'_B(1))^l = b^k (a + s \Delta \rho)^l,
\end{aligned}\end{equation}
where $a=1+\rho_A s>1$, $b=1-\rho_A\frac{s}{1+s}<1$, and $\Delta \rho = \rho_B - \rho_A$.
This leads to a condition similar to eq.~20 in~\citet{Ram2018}, but more complex due to the inclusion of $\Delta \rho$:
\begin{equation} \label{eq:poly_condition_periodic_fluc_k_l}
\frac{-\log{b}}{\log{\big(a + s \Delta \rho \big)}} < 
\frac{l}{k} < 
\frac{\log{a}}{-\log{\big(b - s \Delta \rho/(1+s)\big)}}.
\end{equation}
Therefore, for a given value of $\rho_A$, if $\rho_B>\rho_A$ then increasing the vertical transmission rate of $B$ decreases the environmental period ratio $l/k$ that permits a protected polymorphism; decreasing $\rho_B$  will have the opposite effect: it increases the ratio $l/k$ that permits a protected polymorphism.

% Randomly fluctuating transmission and selection
\subsection*{Randomly fluctuating transmission and selection}

Now suppose that both selection and transmission fluctuate randomly.
Rewrite eq.~28 from \citet{Ram2018} so that $\rho$ is also a random variable
\begin{equation}
x_{t+1} = x_t \frac{1 + z_t + x_t (1 - \rho_t) s_t}{1 + x_t s_t},
\end{equation}
where $z_t=\rho_t s_t$; $s_t$ are i.i.d (independent and identically distributed); $Pr(-1+C<s_t<D)=1$ for some positive $C$ and $D$; $\rho_t$ are i.i.d; and  $0<\rho_t<1$ ($t=0,1,2,\ldots$).
Therefore, $z_t$ are independent and identically distributed and $P(-1+C < z_t < D)$. 

{\it Definition: ``stochastic local stability''.}
A constant equilibrium state $x^*$ is said to be \emph{stochastically locally stable} if for any $\epsilon>0$ there exists a $\delta>0$ such that $|x_0-x^*|<\delta$ implies
\begin{equation}
P(\lim_{t \to \infty}x_t = x^*) \ge 1-\epsilon.
\end{equation}
Thus stochastic local stability holds for $x^*$ provided for any initial $x_0$ sufficiently near $x^*$ the process $x_t$ converges to $x^*$ with high probability.

In our case there are two constant equilibria $x^*=0$ and $x^*=1$ corresponding to fixation in $B$ and $A$, respectively.
From \citet[][results~6 and 7]{Ram2018} we can characterize the stochastic local stability of these fixations as follows.

\begin{itemize}
\item Suppose $E[log(1+z_t)]>0$. Then $x^*=0$ is not stochastically locally stable and in fact $P(lim_{t \to \infty} x_t=0) = 0$, i.e., fixation of $B$ almost surely does not occur.
\item Suppose $E[log(1+z_t)]<0$. Then $x^*=0$ is stochastically locally stable. 
\item Similarly, if $E[log(1-z_t/(1+s_t)]<0$, then $x^*=1$ is stochastically locally stable, and if $E[log(1-z_t/(1+s_t))]>0$, then fixation of $A$ almost surely does not occur.
\item In particular, if $E[z_t] = cov(\rho_t, s_t) + E[\rho_t] E[s_t] \le 0$ then $x^*=0$ is stochastically locally stable, and similarly if $E[-z_t/(1+s_t)] = cov(\rho_t, -s_t/(1+s_t)) - E[\rho_t] E[s_t/(1+s_t)] \le 0$ then $x^*=1$ is stochastically locally stable.
\item It is not possible that $E[log(1+z_t)]$ and $E[log(1-z_t/(1+s_t)]$ are both negative, as their sum is positive:
\begin{equation}\begin{aligned}
{}&E[log(1+z_t)] + E[log(1-z_t/(1+s_t)]  \\
={} &E[log(1+z_t) + log(1-z_t/(1+s_t)] \\
={} &E[log\big((1+z_t)(1-z_t/(1+s_t)\big)] \\
={} &E[log\big( 1+\rho_t(1-\rho_t)s_t^2/(1+s_t) \big)] > 0,
\end{aligned}\end{equation}
and therefore it is not possible that both fixations are stochastically locally stable.
\end{itemize}

\paragraph{Examples.}
First, if $s_t$ and $\rho_t$ are independent ($cov(s_t, \rho_t)=0$) and $s_t$ is symmetric around zero, then $E[-z_t/(1+s_t)] > 0$ (because $E[-z_t/(1+s_t)] = - E[\rho_t] \cdot E[s_t/(1+s_t)]$ and $E[s_t/(1+s_t)] < E[s_t] = 0$).
Therefore, fixation of $B$ is stochastically locally stable and fixation of $A$ almost surely does not occur.
For example, let $s_t \sim U(-1, 1)$ and $\rho_t \sim U(0,1)$ independently (in particular, $cov(\rho_t, s_t)=0$), then $E[log(1+z_t)]\approx -0.07315$ and $E[log(1-z_t/(1+s_t))]\approx 0.2337$.

However, note that symmetry of $s_t$ around zero provides an advantage to phenotype $B$: using Jensen's inequality, $E[w_A/w_B] = E[1+s_t] = 1 \le E[1/(1+s_t)] = E[w_B/w_A]$.
Therefore, if we take the i.i.d fitness random variables for $A$ and $B$ to be $w_{A,t}, w_{B,t} \sim U(0,1)$, respectively, and define $s_t=(w_{A,t}-w_{B,t})/w_{B,t}$, then neither $A$ nor $B$ has an advantage, on average (i.e. $E[w_{A,t}/w_{B,t}]=E[w_{B,t}/w_{A,t}]$, see \autoref{fig:beta}A), and both $E[z_t]$ and $E[-z_t/(1+s_t)]$ are positive, so that both fixations are not stochastically locally stable, and we expect the population to approach a polymorphic distribution.

Second, if $s_t$ and $\rho_t$ are not independent such that $cov(\rho_t, s_t) \ne 0$, a fixation can occur.
Let $w_{A,t}, w_{B,t} \sim U(0,1)$, $s_t=(w_{A,t}-w_{B,t})/w_{B,t}$ and $\rho_t \sim Beta(1+s_t, 1)$ (a beta distribution with parameters $1+s_t$ and $1$). 
The covariance of $s_t$ and $\rho_t$ is positive ($cov(s_t, \rho_t) \approx 4$ as estimated by averaging over $10^8$ random values of $s_t$ and $\rho_t$); that is, vertical transmission is more likely when $A$ is favored (i.e. $s_t>0$) and oblique transmission is more  likely when $B$ is favored (i.e. $s_t<0$; \autoref{fig:beta}).
Then $E[log(1+z_t)] >0$ and $B$ almost surely doesn't fix. 
Also, $E[log(1-z_t / (1+s_t)] < 0$, so fixation of $A$ is stochastically locally stable.
The opposite occurs if $\rho_t \sim \beta(1, 1+s_t)$ and the covariance of $s_t$ and $\rho_t$ is negative ($cov(s_t, \rho_t) \approx -4$). In that case, fixation of $B$ is stochastically locally stable and $A$ almost surely doesn't fix.

% figure
\begin{figure*}[hbt]
\centering
\includegraphics[width=\linewidth]{../figures/{beta}.png}
\caption{
\textbf{Covariance of selection and transmission.}
\textbf{(A)} Histogram of $w_{A,t}/w_{B,t}$ where $w_{A,t}$ and $w_{B,t}$ are identically and independently distributed uniform random variables $U(0,1)$.
\textbf{(B)} Histogram of $s_t = (w_{A,t}-w_{B,t})/w_{B,t}$.
\textbf{(C)} Histogram of $\rho_t \sim Beta(1+s_t, 1)$.
\textbf{(D)} The joint distribution of $\rho_t$ and $s_t$ demonstrates a positive correlation $cov(s_t, \rho_t)>0$.
}
\label{fig:beta}
\end{figure*}

Third, it is also possible that both fixations are not stochastically locally stable even if $s_t$ and $\rho_t$ covary, but, as in the case of periodic fluctuations, this can only occur if fluctuations in $\rho_t$ are small.
For example, \autoref{fig:rho1_rho2_stoch_p} shows the expected outcome when $s_t=s$ and $\rho_t=\rho_1$ with probability $p=0.505$, while $s_t=-s$ and $\rho_t=\rho_2$ with probability $1-p=0.495$.
The blue and red areas show expected fixation of $A$ and $B$, respectively (i.e. stochastic local stability) and the white area shows expected protected polymorphism (i.e. neither fixation is stochastically locally stable). 

% figure
\begin{figure*}[hbt]
\centering
\includegraphics[width=0.75\linewidth]{../figures/{rho1_rho2_stoch_p}.pdf}
\caption{
\textbf{Stochastic local stability.}
Here, $s_t=0.05$ and $\rho_t=\rho_1$ with probability $p=0.505$ and $s_t=-0.05$ and $\rho_t=\rho_2$ with probability $1-p=0.495$.
The diagonal represents the case of no transmission fluctuations; \citet[Fig.~2]{Ram2018} demonstrated that with a constant transmission rate $\rho=0.1$ and the above distribution of $s_t$, neither fixation is stochastically stable.
}
\label{fig:rho1_rho2_stoch_p}
\end{figure*}

% Finite population size
\subsection*{Finite population size}

To include the effects of random genetic drift due to finite population size in the above deterministic model, we follow \citet{Ram2018} and develop a diffusion  approximation.
In~\citet{Ram2018} only selection fluctuated via $s_t$, but here we also have transmission fluctuating via $\rho_t$. 

We obtain a result similar to result 11 from~\citet{Ram2018}:
The mean $\mu(x)$ and variance $\sigma^2(x)$ of the change in the frequency $x$ of phenotype $A$ in the case of a cycling environment $AkBl$, where $k+l=n$, are
\begin{equation} \label{eq:drift_diffusion_terms}
\mu(x) = S_n x(1-x),
\quad \text{and} \quad
\sigma^2(x) = n x (1-x),
\end{equation}
where $S_n = \sum_{t=1}^{n}{z_t}$ and $z_t = \rho_t s_t$.
Furthermore, combining eq.~\ref{eq:drift_diffusion_terms} with eqs.~46-47 from \citet{Ram2018}, we find that the probability of fixation of phenotype $A$ when its initial frequency if $x$, is
\begin{equation}
u(x) = \frac{1 - e^{-2 \frac{S_n}{n} x}}{1 - e^{-2 \frac{S_n}{n}}}.
\end{equation}
From result 10 in \citet{Ram2018}, $u(x)$ is monotone increasing in $S_n/n$ which is the average selection coefficient of $A$ weighted by the vertical transmission rates $\rho_t$.
Therefore, if $s_t$ and $\rho_t$ are positively (negatively) correlated, $S_n/n$ increases (decreases), and the fixation probability $u(x)$ increases (decreases). 
This occurs because selection  affects only those individuals that transmit their phenotype to their own offspring (i.e. vertically), and a fraction $1-\rho_t$ of the population is effectively masked (for better or worse) from selection at each generation.

% Fluctuations in space
\subsection*{Fluctuations in space}

We now describe a model in which fluctuations in selection and transmission occur in space, rather then time.
Consider a population divided to two demes.
Selection (e.g. reproduction) and transmission (e.g. learning, development) occur within the demes, followed by migration of sub-adults---individuals that have already acquired their phenotype but have yet to reproduce.
The frequency of phenotype $A$ in deme $j$ is denoted by $x_j$, and therefore after selection and transmission the frequencies are 
\begin{equation} \label{eq:migration_model_selection_transmission}
x_j^s = \rho_j \frac{w_j}{\overline{w}_j} x_j + (1-\rho_j) x_j,
\end{equation}
where $w_j$ is the fitness of phenotype $A$ in deme $j$ relative to the fitness of phenotype $B$; $\overline{w}_j=w_j x_j + (1-x_j)$ is the mean fitness in deme $j$; and $\rho_j$ is the vertical transmission rate in deme $j$.

Following migration, the frequencies of $A$ in the two demes are
\begin{equation} \label{eq:migration_model_migration}
\begin{aligned}
x_1' &= (1-m_1) x_1^s + m_1 x_2^s, \\
x_2' &= m_2 x_1^s + (1-m_2) x_2^s,
\end{aligned}
\end{equation}
where $0 \le m_1, m_2 \le 1/2$ are the migration rates, such that $m_1$ is the fraction of the population of deme 1 replaced by individuals from deme 2, and vice versa for $m_2$.
This is a \emph{two-deme stepping-stone migration scheme}~\citep[][eq.~2.17]{Karlin1982}.
Analysis of this general model is difficult, though some analytical results may be attained~\citep[see~Principle~6.1]{Karlin1982}.

% Asymmetric migration and symmetric selection
\paragraph{Asymmetric migration and symmetric selection.}

\autoref{fig:asym_migration} shows some numerical results for the case of symmetric selection{\color{red},} $w_1=1/w_2=w>1$.
We focus on composite parameters of the model:
$m_1/m_2$, on the x-axis, is the ratio of migration rates into deme 1 and deme 2; when this ratio is large, deme 1 accepts more migrants than deme 2.
$\rho_1/\rho_2$, on the y-axis{\color{red},} the ratio of the vertical transmission rates in deme 1 and deme 2; when this ratio is large, individuals in deme 1 use vertical transmission more often then individuals in deme 2.

The results demonstrate that fixation of phenotype $A$ is stable if migration to and oblique transmission within deme 1, where $A$ is favored, are higher than in deme 2.
The opposite is true for phenotype $B$.
A protected polymorphism exists if neither fixation is stable: if migration ratios are positively correlated -- vertical transmission occurs more often in the deme that accepts more migrants -- or if both ratios are close to unity, that is, the differences between the demes in terms of migration and transmission are small (\autoref{fig:asym_migration}).

% figure
\begin{figure*}[hbt]
\centering
\includegraphics[width=0.75\linewidth]{../figures/asym_migration.pdf}
\caption{
\textbf{Oblique transmission and asymmetric migration.}
Classification of the stable equilibrium of the system in eqs.~\ref{eq:migration_model_selection_transmission}-\ref{eq:migration_model_migration} for different ratios of the migration rates (x-axis) and vertical transmission rates (y-axis) in the two demes.
Stability was determined for 10,000 random choices of $m_1$, $m_2$, $\rho_1$, and $\rho_2$ by calculating the leading eigenvalue of the Jacobian of the system.
Blue markers denote cases in which the leading eigenvalue of the Jacobian at $x_1=x_2=0$ was less than 1, leading to fixation of $B$.
Red markers denote cases in which the leading eigenvalue of the Jacobian at $x_1=x_2=1$ was less than 1, leading to fixation of $A$.
Green markers denote cases in which both leading eigenvalues were larger than 1, leading to a protected polymorphism.
Here, the fitness values are $w_1=1/w_2=2$.
}
\label{fig:asym_migration}
\end{figure*}

% Symmetric migration and selection
\paragraph{Symmetric migration and selection.}
 
In the case of symmetric migration $m_1=m_2=m$ occurring after symmetric selection $w_1=1/w_2=w>1$, the recursions~\eqref{eq:migration_model_selection_transmission}~and~\eqref{eq:migration_model_migration} become
\begin{equation}\begin{aligned} \label{eq:migration_model_unconditional_symmetric}
x_1' &= (1-m)x_1\Big(\rho_1 \frac{w}{\overline w_1} + 1-\rho_1 \Big) + m x_2\Big(\rho_2 \frac{1/w}{\overline w_2} + 1-\rho_2 \Big), \\
x_2' &= m x_1\Big(\rho_1 \frac{w}{\overline w_1} + 1-\rho_1 \Big) + (1-m) x_2\Big(\rho_2 \frac{1/w}{\overline w_2} + 1-\rho_2 \Big).
\end{aligned}
\end{equation}
This is the \emph{homogeneous stepping-stone migration scheme}~\citep[][eq.~2.14]{Karlin1982}.

We have the following results:
\begin{itemize}
\item With only oblique transmission ($\rho_1=\rho_2=0$), there are only neutral equilibria $(x^*,x^*)$ for any $0 \le x^* \le 1$.
\item With only vertical transmission ($\rho_1=\rho_2=1$), the fixation equilibria  $(0,0)$, $(1,1)$ are unstable and there exists a protected polymorphism
\begin{equation}
\begin{aligned} \label{eq:migration_model_unconditional_symmetric_polymorphism}
x_1^* &= \frac{w-1-m(w+1) + \sqrt{\Delta}}{2(w-1)}, \\
x_2^* &= 1-x_1^*,
\end{aligned}
\end{equation}
where $\Delta = m^2(w+1)^2+(1-2m)(w-1)^2$.
\item With only vertical transmission in one deme ($\rho_1=1$) and a combination of both vertical and oblique transmission in the other deme ($\rho_2=\rho$), fixation of $B$ is unstable, and fixation of $A$ is stable if and only if the vertical transmission rate in deme 2 is below $\hat \rho$; that is,
\begin{equation} \label{eq:migration_model_unconditional_symmetric_condition_rho}
\rho < \hat \rho = \frac{m}{m+(1-m)(w-1)} < 1,
\end{equation}
or if the migration rate is above $\hat m$; that is,
\begin{equation} \label{eq:migration_model_unconditional_symmetric_condition_m}
m > \hat m = \frac{\rho w - 1}{\rho w + 1}.
\end{equation}
\end{itemize}
{\color{red}The proofs of (22) and (23) are in Appendix I.}
\paragraph{Examples.}

Figures \ref{fig:migration_rho} and \ref{fig:migration_m} show the stable frequencies of phenotype $A$~ (eq.~\ref{eq:migration_model_unconditional_symmetric_polymorphism}) and the stable population mean fitnesses in the two demes with symmetric selection where $A$ is favored in deme 1 and $B$ is favored in deme 2 with similar selection intensities.
Notably, in the absence of oblique transmission (\autoref{fig:migration_m}, left column), migration decreases the differences between the demes and reduces the population mean fitnesses.
With some oblique transmission, but equal in both demes, results are similar (not shown).
However, if oblique transmission is stronger in deme 2 than in deme 1 (\autoref{fig:migration_m}, middle and right columns), the stable frequency of $A$ increases in both demes.
Therefore, the mean fitness in deme 1 decreases to a lesser extent than in deme 2, and even increases when the migration rate is high enough.

The polymorphism $(x_1^*, x_2^*)$ (eq.~\ref{eq:migration_model_unconditional_symmetric_polymorphism}) is protected when transmission rates are equal, but not when transmission rates differ enough and migration is strong enough, in which case fixation of phenotype $A$ is stable. 
For example, with enough oblique transmission ($\rho_2<\hat \rho$) in deme 2, phenotype $A$ fixes, and the more migration, the less oblique transmission is required to fix $A$ (see shaded areas in \autoref{fig:migration_rho}).
Similarly, with enough migration ($m > \hat m$), phenotype $A$ fixes, and the more oblique transmission in deme 2, the less migration is needed to fix $A$ (see shaded area in \autoref{fig:migration_m}).

% figure
\begin{figure*}[ht]
\centering
\includegraphics[width=0.75\linewidth]{../figures/migration_rho.pdf}
\caption{
\textbf{Oblique transmission and migration: effect of transmission.} 
The figure shows $x^*_i$ the stable frequencies of $A$ (top row) and $\overline{w}^*_i$ the {\color{red}stable} population mean fitnesses (bottom row) in the two demes.
Selection is symmetric between the two demes (the fitness of phenotype $A$ is $w_1=2$ in deme 1 and $w_2=0.5$ in deme 2; the fitness of phenotype $B$ is $1$ in both demes).
The vertical transmission rate is $\rho_1=1$ in deme 1, and $\rho_2$ (x-axis) in deme 2.
Migration rate $m$ is 0.05, 0.1, or 0.25 in the left, middle, and right columns, respectively.
The shaded area denotes stable fixation of phenotype $A$ according to {\color{red}inequality}~\ref{eq:migration_model_unconditional_symmetric_condition_rho}.
Lines are drawn by iterating eqs.~\ref{eq:migration_model_unconditional_symmetric} until frequencies in consecutive generations differ by less than $10^{-4}$, starting with equal frequencies.
}
\label{fig:migration_rho}
\end{figure*}

% figure
\begin{figure*}[ht]
\centering
\includegraphics[width=0.75\linewidth]{../figures/migration_m.pdf}
\caption{
\textbf{Oblique transmission and migration: effect of migration.} 
The figure shows $x^*_i$, the stable frequencies of $A$ (top row), and $\overline{w}^*_i$, the {\color{red}stable} population mean fitnesses (bottom row), in the two demes.
Selection is symmetric between the two demes (the fitness of phenotype $A$ is $w_1=2$ in deme 1 and $w_2=0.5$ in deme 2; the fitness of phenotype $B$ is $1$ in both demes).
The vertical transmission rate is $\rho_1=1$ in deme 1, and $\rho_2=1$, $0.4$, and $0.2$, in the left, middle, and right columns, respectively, in deme 2.
Migration rate $m$ is on the x-axis.
The shaded area denotes stables fixation of phenotype $A$ according to {\color{red}inequality}~\ref{eq:migration_model_unconditional_symmetric_condition_m}.
Lines are drawn by iterating eq.~\ref{eq:migration_model_unconditional_symmetric} until frequencies in consecutive generations differ by less than $10^{-4}$, starting with equal frequencies.
}
\label{fig:migration_m}
\end{figure*}

% Discussion
\section*{Discussion}

Most models of cultural transmission assume a fixed relative rate at which different modes of transmission---vertical, horizontal, or oblique---occur.
Here we explored a model in which the relative rates of vertical and oblique transmission fluctuate over time, either periodically or randomly, or over space.

Comparing our results with previous results from a similar model with a fixed rate of vertical transmission~\citep{Ram2018}, we find that a protected polymorphism can be maintained only if fluctuations in the rate of vertical transmission are small, and that stronger selection on the transmitted trait permits larger fluctuations in the rate of transmission while still maintaining a protected polymorphism. 
In the case of fluctuations in space, the greater the separation between the two populations (i.e., the smaller the migration rates) the larger can be the fluctuations that maintain polymorphism; however, as migration becomes more frequent, even small differences in the vertical transmission rate will eliminate the polymorphism.
When fluctuations are stochastic, we find that if vertical transmission covaries with selection, the phenotype that has a higher probability of being vertically transmitted  when it is favored will likely eventually fix in the population.
However, if transmission and selection are independent, then a polymorphism can be maintained if selection does not, on average, favor one phenotype over the other.

One caveat of our model is the use of the ``phenotypic gambit''~\citep{Grafen1984}: the assumption that the transmission mode itself is strictly vertically transmitted.
Although there is some evidence that the tendency to use different learning mechanisms is genetically transmitted~\citep{Foucaud2013}, this assumption can be challenged: individuals may be able to learn how and when to learn, in what has been called ``social learning of social learning''~\citep{Mesoudi2016}.
Indeed, it has been demonstrated that guppy fish are more likely to learn from others if previous social experiences provided benefits~\citep{Leris2016}.
It is also possible that the transmission mode is regulated.
For example, \citet{Farine2015} found that zebra fish switch from vertical to oblique transmission after exposure to stress hormones.
Our model accounts for cases in which the entire population changes its transmission mode due to stress, but not for cases in which only {\color{red}specific (e.g. stressed)} individuals do so.

An extension of our model could incorporate more sophisticated oblique transmission schemes~\citep[][Figure~3]{Creanza2017}.
For example, conformity---preference for learning a frequent phenotype---has been demonstrated in wild monkeys~\citep{VanDeWaal2013} and birds~\citep{Aplin2015}.
We suggest that the specific mode of oblique transmission can also fluctuate over time, so that individuals can, for example, conform to the frequent phenotype under benign conditions, and prefer a rare phenotype under stressful conditions.
Additional work will be required to understand how such fluctuations affect the population dynamics. 

% Acknowledgements
{\small
\section*{Acknowledgements}

This work was supported in part by 
the Stanford Center for Computational, Evolutionary and Human Genomics, 
the Morrison Institute for Population and Resources Studies, Stanford University, and the John Templeton Foundation.
}

% bibliography
\bibliographystyle{agsm}
\bibliography{ms_sunnyvale}

% Appendix
\section*{Appendix I}
\paragraph{Proof of (22) and (23).}

%For the stability of the equilibria when $\rho_1=\rho_2=1$, the characteristic polynomial of the Jacobian is the same for both fixation equilibria $(0,0)$ and $(1,1)$ and is given by $f(\lambda) = a \lambda^2 + b \lambda + c$ with
%\begin{equation}
%\begin{aligned}
%a &= 1, \quad
%b &= -\frac{(1-m)(w^2+1)}{w}, \quad
%c &=  1-2m.
%\end{aligned}
%\end{equation}
%First, the discriminant $\Delta = b^2-4ac$ is positive for any real $m$, because, when writing it as a quadratic polynomial in $m$, $\Delta(m)=\frac{1}{w^{2}} \big(m^{2} (w^2+1)^2 - 2 m (w^2-1)^2 + (w^2-1)^2\big)$, this polynomial is positive at $m=0$ and $m=1/2$ and has a negative discriminant $\big(-16w^2(w^2-1)^2\big)$.
%Therefore, the discriminant $\Delta$ is positive for any $0 \le m \le 1/2$ and $f(\lambda)$ has two positive roots. \\
%Second, as $a>0$, $b<0$, and $c>0$, and because $a+b+c=-\frac{1-m}{w}(w-1)^2<0$, the leading eigenvalue is greater than one, both equilibria are unstable, and a protected polymorphism exists and is given by~\eqref{eq:migration_model_unconditional_symmetric_polymorphism}.

When $\rho_1=1$ and $\rho_2=\rho$, with $0 \le \rho \le 1$, the stability of $(0,0)$ (i.e. fixation in $B$) is determined by the characteristic polynomial of the Jacobian of~\eqref{eq:migration_model_unconditional_symmetric}, $f_0(x)=ax^2+bx+c$, with coefficients
\begin{equation}
\begin{aligned}
a &= 1, \quad
b &= -(1 - m) (w + 1 - \rho + \rho/w), \quad
c &=  (1 - 2m)(\rho + w(1-\rho)).
\end{aligned}
\end{equation}

First, the discriminant $\Delta = b^2-4ac$ for $m=0$ and $m=1/2$ is
\begin{equation}\begin{aligned}
&\Delta(m=0) = \Big(\frac{(w + \rho) (w-1)}{w}\Big)^2, \quad \text{and} \\
&\Delta(m=1/2) = \Big(\frac{w^2 + w(1-\rho) + \rho}{2 w}\Big)^2,
\end{aligned}\end{equation}
which are both positive.
Writing $\Delta$ as a polynomial in $m$, $g(m)$, the discriminant of $g(m)$ is 
$$
\big(- 16 (w + \rho)^2 (w - 1)^2 \big(w (1 - \rho) + \rho \big) / w^2\big),
$$
which is negative, and therefore the discriminant $\Delta$ of $f_0(x)$ is positive for any $0 \le m \le 1/2$ and $f_0(x)$ has two real roots. \\
Second, $a>0$, $b<0$, and $c>0$ so $f_0$ is positive with a negative derivative at $x=0$ and  a positive derivative at infinity.
Therefore stability of $(0,0)$ can be determined by the sign of $f_0(1)=a+b+c$.
For $\rho=1$ we have $a+b+c=-\frac{1-m}{w}(w-1)^2<0$.
For $\rho=0$ we have $a+b+c=-m(w-1)<0$.
Finally, $a+b+c$ is a linear function of $\rho$ and therefore $a+b+c<0$ for any $0 \le \rho \le 1$, so $f_0(x)$ has a real root greater than one, and $(0,0)$ is unstable.

The stability of $(1,1)$ (i.e. fixation in $A$) is determined by the characteristic polynomial $f_1(x)=ax^2+bx+c$ with coefficients
\begin{equation}
\begin{aligned}
a &= 1, \quad
b &= -(1 - m) (w\rho + 1 - \rho + 1/w), \quad
c &=  (1 - 2m)(\rho + (1-\rho)/w).
\end{aligned}
\end{equation}

First, the discriminant $\Delta$ of $f_1(x)=0$ is $b^2-4ac$ for $m=0$ and $m=1/2$ is
\begin{equation}\begin{aligned}
&\Delta(m=0) = \Big(\frac{(w \rho + 1) (w-1)}{w}\Big)^2, \quad \text{and} \\
&\Delta(m=1/2) = \Big(\frac{w^2 \rho + w(1-\rho) + 1}{2 w}\Big)^2,
\end{aligned}\end{equation}
which are both positive. 
Writing the discriminant $\Delta$ of $f_1(x)$ as a polynomial in $m$, $g(m)$, the discriminant of $g(m)$ is
$$
\big(- 16 (w\rho + 1)^2 (w - 1)^2 \big(\rho w + 1 - \rho \big) / w^3\big),
$$
which is negative, and therefore $\Delta$ is positive for any $0 \le m \le 1/2$ and $f_1(x)$ has two real roots. \\
Second, $a>0$, $b<0$, and $c>0$ so $f_1$ is positive with a negative derivative at $x=0$ and  a positive derivative at infinity.
Therefore, fixation of $A$ is stable {\color{red}if and only if} both $f_1(1)=a+b+c$ and $f_1'(1)=2a+b$ are positive.
{\color{red}Now $f_1(1)>0$  if $\rho < \frac{m}{m+(1-m)(w-1)}=M_1$, and
$f_1'(1)>0$  if $\rho < \frac{m(w+1)+w-1}{(1-m)w(w-1)}=M_2$.
The difference $M_1-M_2$ is a quadratic in $w$ with a negative discriminant $2m-1$ and negative value $-1/2$ at $(w=2, m=0)$, so $M_1<M_2$ for any $w>1$ and $0 \le m \le 1/2$.}

Therefore, fixation of $A$ is stable if {\color{red}$\rho < M_1 = \frac{m}{m+(1-m)(w-1)}$}~\eqref{eq:migration_model_unconditional_symmetric_condition_rho}. By rearranging this inequality, we can also obtain an expression for $m$~\eqref{eq:migration_model_unconditional_symmetric_condition_m}.


\end{document}